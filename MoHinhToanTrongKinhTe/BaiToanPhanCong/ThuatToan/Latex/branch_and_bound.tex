\section{Thuật toán Nhánh và Cận (Branch and Bound - B\&B)}

Thuật toán Nhánh và Cận là một phương pháp tìm kiếm có hệ thống và hiệu quả để giải các bài toán tối ưu hóa tổ hợp, như bài toán phân công, bài toán balô, bài toán du lịch thương gia, v.v. Mục tiêu của thuật toán là khám phá không gian nghiệm một cách có tổ chức và loại bỏ sớm các nhánh không có khả năng chứa nghiệm tối ưu, từ đó giảm đáng kể khối lượng tính toán.

\subsection{Ý tưởng chính của thuật toán}

Thuật toán Nhánh và Cận hoạt động bằng cách phân chia không gian nghiệm thành các tập con (nhánh) và xác định cận dưới cho giá trị tối ưu của bài toán trong từng nhánh. Những nhánh có cận dưới cao hơn giá trị tốt nhất tìm được cho đến thời điểm hiện tại sẽ bị loại bỏ (cắt tỉa). Những nhánh còn lại sẽ tiếp tục được phân chia và kiểm tra cho đến khi tìm được lời giải tối ưu.

\subsection{Các khái niệm cơ bản}

\begin{itemize}
    \item \textbf{Nhánh}: Một nút trong cây tìm kiếm của thuật toán, đại diện cho một trạng thái tạm thời của bài toán (tức là một phần của lời giải).
    \item \textbf{Cận dưới}: Giá trị ước lượng nhỏ nhất của lời giải có thể đạt được trong một nhánh. Cận này giúp quyết định xem có nên tiếp tục mở rộng nhánh hay không.
    \item \textbf{Cận trên}: Giá trị tốt nhất của lời giải tìm được đến thời điểm hiện tại. Cận trên giúp so sánh với cận dưới để loại bỏ các nhánh không có triển vọng.
\end{itemize}

\subsection{Các bước của thuật toán Nhánh và Cận}

Thuật toán Nhánh và Cận được thực hiện qua các bước sau:

\subsubsection*{Bước 1: Khởi tạo và tính toán cận dưới ban đầu}
Bắt đầu với một lời giải tạm thời chưa hoàn chỉnh (chưa có gán nào được thực hiện). Tính toán cận dưới ban đầu để biết được giá trị thấp nhất có thể đạt được. Cách đơn giản là tính cận dưới bằng cách giảm hàng hoặc giảm cột của ma trận chi phí, tức là trừ đi giá trị nhỏ nhất của mỗi hàng hoặc mỗi cột khỏi tất cả các phần tử trong hàng hoặc cột đó. Tổng các giá trị trừ đi sẽ là cận dưới.

\subsubsection*{Bước 2: Phân nhánh}
Dựa trên lời giải tạm thời hiện tại, chọn một người và gán cho họ một công việc (hoặc chọn một biến khác tuỳ theo bài toán). Mỗi lần gán sẽ tạo ra một nhánh mới trong cây tìm kiếm. Mỗi nhánh sẽ đại diện cho một lựa chọn gán khác nhau và có một cận dưới riêng biệt.

\subsubsection*{Bước 3: Tính cận dưới mới}
Sau mỗi lần gán, cần tính lại cận dưới cho phần còn lại của bài toán (tức là phần chưa được gán). Cận dưới này cho biết lời giải tốt nhất có thể đạt được nếu tiếp tục mở rộng nhánh này. Nếu cận dưới của một nhánh cao hơn cận trên hiện tại (giá trị của lời giải tốt nhất đã tìm được), nhánh đó sẽ bị cắt tỉa, nghĩa là không cần mở rộng thêm.

\subsubsection*{Bước 4: Cắt tỉa (pruning)}
Những nhánh có cận dưới cao hơn cận trên sẽ bị loại bỏ (cắt tỉa), vì chúng không thể chứa lời giải tối ưu. Điều này giúp giảm số lượng các nhánh cần phải khám phá.

\subsubsection*{Bước 5: Cập nhật lời giải tốt nhất}
Mỗi khi tìm được một lời giải hoàn chỉnh (tất cả các biến đã được gán), nếu tổng chi phí của lời giải này thấp hơn cận trên hiện tại, cập nhật cận trên với giá trị mới này. Lời giải này tạm thời là lời giải tốt nhất cho đến khi có lời giải khác tốt hơn.

\subsubsection*{Bước 6: Lặp lại quá trình}
Tiếp tục phân nhánh và cắt tỉa cho đến khi tất cả các nhánh đã được khám phá hoặc cắt tỉa. Khi không còn nhánh nào khả thi, lời giải tối ưu là lời giải tốt nhất đã được tìm thấy trong quá trình.

\subsection{Ví dụ minh họa}

Giả sử ta có một ma trận chi phí cho bài toán phân công:

\[
\begin{bmatrix}
8 & 2 & 5 \\
3 & 2 & 7 \\
4 & 1 & 6 \\
\end{bmatrix}
\]

\begin{itemize}
    \item \textbf{Bước 1}: Tính cận dưới ban đầu bằng cách giảm hàng. Chọn giá trị nhỏ nhất trong mỗi hàng:
    \[
    [8-2, 2-2, 5-2] = [6, 0, 3]
    \]
    \[
    [3-2, 2-2, 7-2] = [1, 0, 5]
    \]
    \[
    [4-1, 1-1, 6-1] = [3, 0, 5]
    \]
    Cận dưới ban đầu là tổng các giá trị đã trừ \(2 + 2 + 1 = 5\).
    
    \item \textbf{Bước 2}: Gán Người 3 cho Công việc 2 (chi phí = 0).
    Loại bỏ hàng 3 và cột 2 khỏi ma trận. Ma trận còn lại:
    \[
    \begin{bmatrix}
    6 & 3 \\
    1 & 5 \\
    \end{bmatrix}
    \]
    
    \item \textbf{Bước 3}: Tính lại cận dưới mới. Sau khi trừ giá trị nhỏ nhất của từng hàng:
    \[
    [6-3, 3-3] = [3, 0]
    \]
    \[
    [1-1, 5-1] = [0, 4]
    \]
    Cận dưới mới là \(1 + 3 + 5 = 9\).
    
    \item \textbf{Bước 4}: Gán Người 2 cho Công việc 1 (chi phí = 0). Ma trận còn lại:
    \[
    \begin{bmatrix}
    0 \\
    \end{bmatrix}
    \]
    
    \item \textbf{Bước 5}: Gán Người 1 cho Công việc 3 (chi phí = 0). Tổng chi phí của lời giải là \(5 + 3 + 1 = 9\). Đây là lời giải tối ưu vì không có nhánh nào có cận dưới thấp hơn.
\end{itemize}

\subsection{Kết luận}

Phương pháp Nhánh và Cận là một cách tiếp cận mạnh mẽ và có hệ thống để giải các bài toán tối ưu hóa tổ hợp. Nó đảm bảo tìm được lời giải tối ưu thông qua quá trình phân nhánh và cắt tỉa các phần không cần thiết của không gian nghiệm, giúp giảm khối lượng tính toán đáng kể so với việc kiểm tra mọi khả năng. Tuy nhiên, thuật toán này có thể tốn nhiều tài nguyên tính toán với những bài toán có kích thước lớn nếu không có chiến lược cắt tỉa tốt.
