\section{Phần Lý Thuyết}
\subsection* {Câu 3} Hãy nêu những sự khác biệt giữa tri thức kinh nghiệm và tri
thức khoa học\\
{\bf Trả lời}
\vspace*{0.5cm}

\begin{tabular}{|m{7cm}|m{7cm}|}
    \hline
    \textbf{Tri thức kinh nghiệm} &
    \textbf{Tri thức khoa học} \\
    \hline
    \begin{itemize}
        \item Dựa trên kinh nghiệm cá nhân: Tri thức này phản ánh những kinh nghiệm cá nhân, những bài học được học từ thực tế cuộc sống, công việc và môi trường xã hội.
        \item Phụ thuộc vào cá nhân: Tri thức kinh nghiệm thường mang tính cá nhân hóa cao, được hình thành qua những trải nghiệm và góc nhìn riêng của mỗi người.
        \item Không qua quá trình nghiên cứu hệ thống: Tri thức này thường không được thẩm định hoặc xác nhận thông qua quá trình nghiên cứu khoa học.
    \end{itemize} &
    \begin{itemize}
        \item Dựa trên phương pháp nghiên cứu khoa học: Tri thức khoa học được xây dựng thông qua việc áp dụng phương pháp khoa học, bao gồm quan sát, thử nghiệm, và phân tích dữ liệu.
        \item Chủ yếu dựa trên sự kiểm chứng: Tri thức khoa học được xác nhận thông qua quá trình kiểm chứng và lặp lại, giúp nó trở nên đáng tin cậy và khách quan hơn.
        \item Phản ánh sự tiến bộ của kiến thức: Tri thức khoa học thường được cập nhật và điều chỉnh dựa trên các phát hiện mới và nghiên cứu tiến bộ.
    \end{itemize} \\
    \hline
    \end{tabular}
\vspace*{0.5cm}
\subsection*{Câu 4} Mục đích và mục tiêu nghiên cứu có phải là những khái niệm
đồng nhất không? Hãy minh họa cho lập luận của mình thông qua một ví
dụ cụ thể nào đó.\\
{\bf Trả lời}\\
Mục đích nghiên cứu là mục tiêu nghiên cứu là hai khái niệm không đồng nhất.\\
{\bf Mục đích nghiên cứu}: Mục đích nghiên cứu là lý do chính để tiến hành nghiên cứu, thường là một tuyên bố tổng quát về ý nghĩa và ý kiến của việc nghiên cứu trong lĩnh vực cụ thể. Mục đích có thể là việc mở rộng kiến thức, giải quyết vấn đề cụ thể, đóng góp vào lý thuyết, hoặc áp dụng kiến thức vào thực tiễn.\\
{\bf Mục tiêu nghiên cứu}: Mục tiêu nghiên cứu là các mục tiêu cụ thể và đo lường được đề ra để đạt được mục đích nghiên cứu. Mục tiêu có thể là việc thu thập dữ liệu, phân tích kết quả, kiểm chứng giả định, hoặc tạo ra một sản phẩm cụ thể.\\
{\bf Ví dụ:}\\
Đề tài: “Nghiên cứu ảnh hưởng của công nghệ 4.0 đến hoạt động sản xuất công nghiệp ở Việt Nam”\\
{\bf Mục đích của đề tài:} Tăng cường hiệu quả và năng suất trong hoạt động sản xuất công nghiệp ở Việt Nam.\\
{\bf Mục tiêu của đề tài:} Tìm ra những ảnh hưởng của công nghệ 4.0 đến quy trình sản xuất công nghiệp để kịp thời áp dụng và cải tiến các phương pháp sản xuất hiện đại.
\subsection*{Câu 5} Những phẩm chất nào cần được đặc biệt đề cao đối với một người làm công tác nghiên cứu khoa học?
\subsection*{Trả lời}
Một trong những phẩm chất quan trọng của người làm công tác nghiên cứu khoa học là tính trung thực và đạo đức nghề nghiệp.
Người làm khoa học không được gian lận trong tất cả các bước của việc nghiên cứu,
trung thực trong việc lựa chọn mẫu dữ liệu,
trung thực trong việc đưa ra kết quả thực nghiệm.
Kết quả nghiên cứu phải được báo cáo một cách chính xác và đầy đủ, bao gồm cả những dữ liệu không phù hợp với giả thuyết ban đầu.
Tránh việc sửa đổi, làm sai lệch hoặc làm đẹp dữ liệu để phù hợp với kết quả mong muốn. Tính trung thực tạo nền tảng cho sự tiến bộ và phát triển của khoa học, vì các nghiên cứu dựa trên những kết quả trung thực mới có thể xây dựng và phát triển tiếp.
\subsection*{Câu 6} Phân tích vai trò của khâu chọn đề tài trong nghiên cứu khoa
học.
\subsection*{Trả lời}
Việc chọn đề tài trong nghiên cứu khoa học giúp xác định được hướng đi và mục tiêu cụ thể của việc nghiên cứu, tạo nền tảng cho các bước tiếp theo như phương pháp và phân tích.
Việc lựa chọn đúng đề tài phù hợp với người làm nghiên cứu khoa học giúp duy trì đam mê và nhiệt huyết, tạo động lực để hoàn thành công việc của nhà nghiên cứu.
Chọn đề tài để đảm bảo tính khả thi về nguồn lực và thời gian để có thể hoàn thành được công việc.
Chọn đề tài phù hợp sẽ đảm bảo được tính khả thi cho cơ sở khoa học và tính hợp lý để đảm bảo nghiên cứu có giá trị và đáng tin cậy. 
Khâu chọn đề tài là bước khởi đầu quan trọng, quyết định thành công và giá trị của nghiên cứu khoa học. Lựa chọn đề tài đúng không chỉ đảm bảo nghiên cứu có tính khả thi và hợp lý mà còn đóng góp tích cực cho sự phát triển của khoa học và xã hội.
\subsection*{Câu 7} 
Hãy bình luận về vai trò của tạp chí Mathematical Reviews
đối với những người làm nghiên cứu toán.
\subsection*{Trả lời}
Mathematical Reviews là một tạp chí khoa học hàng đầu trong lĩnh vực toán học, được xuất bản bởi Hội Toán học Hoa Kỳ từ năm 1940,  có phiên bản điện tử là Mathscinet.
Mathematical Reviews bao gồm các bài đánh giá và tóm tắt về nhiều lĩnh vực khác nhau trong toán học, từ toán lý thuyết đến ứng dụng.
Tạp chí này chứa thông tin của 3 triệu bài báo, 1800 tạp chí của hơn 10000 tác giả, có nhiều ngôn ngữ nhưng ngôn ngữ Anh là chính.
Hàng năm, hơn 100,000 mục mới được thêm vào cơ sở dữ liệu của Mathematical Reviews.\\
Mathematical Reviews có vai trò quan trọng đối với những người làm nghiên cứu toán. Chúng giúp các nhà nghiên cứu tiết kiệm thời gian bằng cách cung cấp tóm tắt và đánh giá về các bài báo, giúp họ nhanh chóng tìm thấy tài liệu phù hợp mà không phải lục tìm qua vô số tạp chí chuyên ngành. Những tóm tắt này cũng giúp các nhà khoa học quyết định có nên đọc chi tiết một bài báo dài hay không, tiết kiệm thời gian để họ có thể tập trung vào các công việc quan trọng khác. Đồng thời, các tạp chí này đảm bảo rằng các bài báo khoa học sẽ đến được đúng đối tượng độc giả, thúc đẩy nghiên cứu và phát triển khoa học.
\subsection*{Câu 8}
Impact factor của một bài báo khoa học là gì?
\subsection*{Trả lời}
Impact factor - IF là một số đo phản ánh số lượng trích dẫn trung bình theo năm của các bài toán khoa học được xuất bản gần đây trên tạp chí đó.
Nó phản ánh mức độ ảnh hưởng tương đối của một tạp chí trong cùng lĩnh vực chuyên ngành, tạp chí có IF cao thường được coi là có tầm quan trọng hơn so với tạp chí có IF thấp.\\
Công thức tính IF trong năm $y$ bất kỳ là tổng số lần trích dẫn của các bài báo được xuất bản trong hai năm trước đó, chia cho tổng số bài báo được xuất bản trong hai năm đó. Cụ thể:
\[ 
    IF_y = \frac{\text{Số trích dẫn}_{y-1} + \text{Số trích dẫn}_{y-2}}{\text{Số bài báo}_{y-1} + \text{Số bài báo}_{y-2}}
\]
\subsection*{Câu 9}
Hãy cho biết mã số chuyên ngành mà anh, chị đang theo học.
\subsection*{Trả lời}
Ngành: Lí thuyết xác suất và thống kê toán học\\
% Tên ngành tiếng anh: Probability Theory and Mathematical Statistics\\
% Mã số (từ năm 2013 đến năm 2017): 60460106\\
Mã số: 8460106
\subsection*{Câu 10}
Dựa theo những văn bản hướng dẫn của trường Đại học
KHTN, hãy trình bày một luận văn thạc sĩ toán học (trừ phần nội dung).
Tất cả các thông tin về bản luận văn đều có thể là giả định.
\subsection*{Trả lời}
\newpage
\includepdf[pages=-]{luanvan.pdf}