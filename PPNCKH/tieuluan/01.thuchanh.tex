\section{Phần thực hành}
{\bf Câu 1.} Hãy làm một bản tóm tắt khoa học và một bản nhận xét khoa
học đối với một bài báo nào đó (do học viên tự chọn) đã được đăng trên các
tạp chí ngành Toán (cần đính kèm bản copy bài báo).\\
{\bf Trả lời}\\
Bài báo đính kèm: \hyperref[appendix:a]{\bf A STUDY AND PERFORMANCE ANALYSIS OF RSA ALGORITHM}.\\
{\bf Tóm tắt khoa học:} Bài báo "A Study and Performance Analysis of RSA Algorithm" của M. Preetha và M. Nithya phân tích hiệu suất của thuật toán RSA và phiên bản cải tiến RSA-OAEP.
Thuật toán RSA là thuật toán mã hóa khóa công khai phổ biến, dựa trên độ khó của việc phân tích nhân tử các số nguyên lớn.
Bài báo so sánh RSA và RSA-OAEP qua phân tích thời gian mã hóa và giải mã trên nhiều kích thước tệp khác nhau.
Kết quả cho thấy RSA-OAEP mất nhiều thời gian hơn nhưng cung cấp mức độ bảo mật cao hơn so với RSA tiêu chuẩn.\\
{\bf Nhận xét khoa học:} Bài báo này cung cấp một phân tích chi tiết về hiệu suất của thuật toán RSA và phiên bản nâng cao RSA-OAEP. Một điểm nổi bật của bài báo là sự kết hợp giữa lý thuyết và thực nghiệm, làm rõ thời gian mã hóa và giải mã cho các kích thước tệp khác nhau. Kết quả cho thấy RSA-OAEP mặc dù tốn nhiều thời gian hơn, nhưng lại cung cấp mức độ bảo mật cao hơn nhờ lớp đệm ngẫu nhiên bổ sung.\\
Bài báo đặc biệt nhấn mạnh việc sử dụng OAEP để tăng cường bảo mật cho RSA, giúp làm tăng độ khó cho các cuộc tấn công mã hóa. Tuy nhiên, sự cân bằng giữa bảo mật và hiệu suất cần được cân nhắc kỹ lưỡng trong các ứng dụng thực tế.\\
Dù nghiên cứu đã đưa ra một so sánh rõ ràng giữa RSA và RSA-OAEP, bài báo có thể được cải thiện bằng cách mở rộng phân tích sang các thuật toán khác như ECC (Elliptic Curve Cryptography) để cung cấp một cái nhìn toàn diện hơn. Hơn nữa, thử nghiệm trên nhiều loại dữ liệu và môi trường khác nhau sẽ giúp tăng tính tổng quát của kết quả.\\
Tóm lại, bài báo đóng góp quan trọng vào lĩnh vực mã hóa bằng cách cung cấp các phân tích và kết quả thực nghiệm chi tiết, giúp người đọc hiểu rõ hơn về hiệu suất và bảo mật của các thuật toán RSA và RSA-OAEP. Tuy nhiên, để đảm bảo tính tổng quát, cần thêm các so sánh với các thuật toán khác và các thử nghiệm trong môi trường đa dạng hơn.\\
{\bf Câu 2. } Hãy chọn một chủ đề bất kỳ trong chương trình toán cao cấp
để soạn thành một bài giảng theo phong cách viết giáo trình.\\
Chủ đề: Ma trận - Định thức\\
Bài giảng: Trang sau
\newpage
\begin{center}
    \huge {\bf MA TRẬN - ĐỊNH THỨC}
\end{center}

\subsection*{1. Ma trận}
\subsubsection*{1.1. Định nghĩa ma trận}
Một bảng số hình chữ nhật gồm có $m$ dòng (hàng) và $n$ cột được gọi là ma trận có cấp (cỡ) $m \times n$.

\begin{center}
Ký hiệu: $A = \begin{pmatrix}
a_{11} & a_{12} & \cdots & a_{1n} \\
a_{21} & a_{22} & \cdots & a_{2n} \\
\vdots & \vdots & \ddots & \vdots \\
a_{m1} & a_{m2} & \cdots & a_{mn}
\end{pmatrix} = \left( a_{ij} \right)_{m \times n} \quad \text{(1.1)}$

\end{center}

\vspace{10pt}
với
\begin{itemize}
    \item $i$: gọi là chỉ số dòng (hàng).
    \item $j$: gọi là chỉ số cột.
    \item $a_{ij}$: là phần tử nằm ở dòng $i$ và cột $j$ trong ma trận $A$.
\end{itemize}

\textbf{Ví dụ 1.} Cho các ma trận
\[
\begin{array}{ll}
A = \begin{pmatrix}
1 & 2 & 3 \\
4 & 5 & 6
\end{pmatrix} & \text{là ma trận cấp } (2 \times 3). \\
B = \begin{pmatrix}
1 & 4 \\
2 & 5 \\
3 & 6
\end{pmatrix} & \text{là ma trận cấp } (3 \times 2). \\
C = \begin{pmatrix}
1 & 3 & 2 \\
0 & 1 & 3 \\
2 & 4 & 5
\end{pmatrix} & \text{là ma trận cấp } (3 \times 3) (\text{ma trận vuông cấp } 3).
\end{array}
\]

\subsubsection*{1.2. Ma trận bằng nhau}
Hai ma trận được gọi là bằng nhau nếu chúng cùng cấp và có tất cả các phần tử tương ứng vị trí bằng nhau.

Cho hai ma trận: \( A = (a_{ij})_{m \times n} \) và \( B = (b_{ij})_{m \times n} \)

\[
A = B \Leftrightarrow \begin{cases}
a_{ij} = b_{ij} \\
\forall i = 1, 2, ..., m; \quad \forall j = 1, 2, ..., n
\end{cases} \quad (1.2)
\]
\subsubsection*{Ví dụ 2.}
Cho hai ma trận: \( A = \begin{pmatrix}
1 & -2 \\
3 & -4
\end{pmatrix} \);
\( B = \begin{pmatrix}
1 & b \\
a & -4
\end{pmatrix} \).
Tìm \( a, b \) để hai ma trận \( A \), \( B \) bằng nhau.

\textit{Giải}

Ta có hai ma trận \( A \) và \( B \) đều có cấp là \( (2 \times 2) \). Do đó \( A = B \Leftrightarrow \begin{cases}
a = 3 \\
b = -2
\end{cases} \).

\subsubsection*{1.3. Các ma trận đặc biệt}
\subsubsection*{1.3.1. Ma trận không}
Ma trận không là ma trận mà các phần tử đều là số không.

\textbf{Ví dụ 3.}
Cho các ma trận không
\[
O_{2 \times 3} = \begin{pmatrix}
0 & 0 & 0 \\
0 & 0 & 0
\end{pmatrix} \quad \text{là ma trận không cấp } (2 \times 3).
\]
\[
O_{3 \times 2} = \begin{pmatrix}
0 & 0 \\
0 & 0 \\
0 & 0
\end{pmatrix} \quad \text{là ma trận không cấp } (3 \times 2).
\]
\subsubsection*{1.3.2. Ma trận vuông}
Ma trận vuông là ma trận có số hàng và số cột bằng nhau. Ma trận vuông cấp \( n \times n \) được gọi tắt là ma trận vuông cấp \( n \). Tập hợp tất cả các ma trận vuông cấp \( n \) được ký hiệu là \( M_n \). Với ma trận vuông \( A \in M_n \), các phần tử \( a_{11}, a_{22}, \ldots, a_{nn} \) được gọi là thuộc \textit{đường chéo (chính)} của ma trận \( A \). Các phần tử \( a_{n1}, a_{n-1,2}, \ldots, a_{1n} \) được gọi là thuộc \textit{đường chéo phụ} của ma trận \( A \).

\textbf{Ví dụ 4.} Cho ma trận vuông cấp 3:
\[
A = \begin{pmatrix}
1 & 2 & 3 \\
4 & 5 & 6 \\
7 & 8 & 9
\end{pmatrix}
\]
có các phần tử \( a_{11} = 1, a_{22} = 5, a_{33} = 9 \) thuộc đường chéo chính còn các phần tử \( a_{31} = 7, a_{22} = 5, a_{13} = 3 \) thuộc đường chéo phụ.

\subsubsection*{1.3.3. Ma trận chéo}
Ma trận chéo là ma trận vuông mà mọi phần tử không thuộc đường chéo chính đều là bằng 0.

\textbf{Ví dụ 5.} Cho ma trận chéo cấp 3:
\[
A = \begin{pmatrix}
1 & 0 & 0 \\
0 & 5 & 0 \\
0 & 0 & 9
\end{pmatrix}
\]
\subsubsection*{1.3.4. Ma trận đơn vị cấp}
Ma trận đơn vị là ma trận chéo mà mọi phần tử thuộc đường chéo chính đều bằng 1. Ký hiệu \( I_n \) là ma trận đơn vị cấp \( n \).

\textbf{Ví dụ 6.} Cho các ma trận đơn vị
\[
I_2 = \begin{pmatrix}
1 & 0 \\
0 & 1
\end{pmatrix}
; \quad
I_3 = \begin{pmatrix}
1 & 0 & 0 \\
0 & 1 & 0 \\
0 & 0 & 1
\end{pmatrix}
; \quad \ldots; \quad
I_n = \begin{pmatrix}
1 & 0 & \ldots & 0 \\
0 & 1 & 0 & \ldots \\
\vdots & \vdots & \ddots & \vdots \\
0 & 0 & \ldots & 1
\end{pmatrix}
.
\]

\subsubsection*{1.3.5. Ma trận tam giác trên (dưới)}
Ma trận tam giác trên (dưới) là ma trận vuông mà các phần tử ở phía dưới (hoặc ở phía trên) đường chéo chính đều bằng 0.

\textbf{Ví dụ 7.} Cho các ma trận cấp 3
\[
\begin{pmatrix}
1 & 2 & 3 \\
0 & 4 & 5 \\
0 & 0 & 6
\end{pmatrix}
\quad \text{là ma trận tam giác trên.}
\]
\[
\begin{pmatrix}
1 & 0 & 0 \\
2 & 4 & 0 \\
5 & 4 & 3
\end{pmatrix}
\quad \text{là ma trận tam giác dưới.}
\]

\subsubsection*{1.3.6. Ma trận bậc thang (ma trận hình thang)}
Ma trận bậc thang là ma trận ứng với hai dòng bất kỳ số hạng khác không đầu tiên của hàng dưới phải nằm bên phải số hạng khác không đầu tiên của hàng trên.
\[
\begin{pmatrix}
a_{11} & a_{12} & \cdots & a_{1r} & \cdots & a_{1n} \\
0 & a_{22} & \cdots & a_{2r} & \cdots & a_{2n} \\
\vdots & \vdots & \ddots & \vdots & \ddots & \vdots \\
0 & 0 & \cdots & a_{rr} & \cdots & a_{rn} \\
0 & 0 & \cdots & 0 & \cdots & 0 \\
\vdots & \vdots & \ddots & \vdots & \ddots & \vdots \\
0 & 0 & \cdots & 0 & \cdots & 0
\end{pmatrix}
\]
với \( r < n \) và \( a_{11}, a_{22}, \ldots, a_{rr} \) gọi là các phần tử chéo.

\textbf{Ví dụ 8.} Cho ma trận bậc thang như sau:
\[
\begin{pmatrix}
1 & 2 & 3 & 4 & 5 \\
0 & 2 & 4 & 3 & 7 \\
0 & 0 & 3 & 5 & 4 \\
0 & 0 & 0 & 5 & 8
\end{pmatrix}
\]
\textit{Lưu ý:} Ma trận tam giác trên là ma trận bậc thang đặc biệt.

\subsubsection*{1.3.7. Ma trận chuyển vị}
Cho \( A = (a_{ij})_{m \times n} \in M_{m \times n} \), chuyển vị của \( A \), ký hiệu \( A^T \), là ma trận cấp \( n \times m \) xác định bởi \( A^T = (a_{ji})_{n \times m} \in M_{n \times m} \).

\textit{Nhận xét:} Ma trận chuyển vị của \( A \) là ma trận nhận được từ \( A \) bằng cách chuyển hàng của \( A \) thành cột của \( A^T \).

\textit{Tính chất}
\begin{itemize}
    \item[(i)] \( (A^T)^T = A \),
    \item[(ii)] \( (A + B)^T = A^T + B^T \),
    \item[(iii)] \( (AB)^T = B^T A^T \).
\end{itemize}

\textit{Định nghĩa:} Ma trận vuông \( A \) được gọi là một \textit{ma trận đối xứng} nếu \( A = A^T \).

\textbf{Ví dụ 9.} Cho ma trận
\[
A = \begin{pmatrix}
2 & 3 & 4 \\
4 & 5 & 6
\end{pmatrix}
\quad \text{là ma trận cấp } (2 \times 3).
\]
Ta có
\[
A^T = \begin{pmatrix}
2 & 4 \\
3 & 5 \\
4 & 6
\end{pmatrix}
\quad \text{là ma trận chuyển vị của ma trận } A \text{ có cấp là } (3 \times 2).
\]

\subsection*{1.4. Các phép toán trên ma trận}
\subsubsection*{1.4.1. Nhân một số thực với ma trận}
Nhân số thức với ma trận là nhân số đó với tất cả các phần tử của ma trận:

Cho ma trận \( A = (a_{ij})_{m \times n} \) và \( \forall k \in \mathbb{R} \) ta có:
\[
kA = (k \cdot a_{ij})_{m \times n}
\]

Đặc biệt \( (-1)A = -A = (-a_{ij})_{m \times n} \quad (1.3) \)


\subsubsection*{1.4.2. Cộng hai ma trận cùng cấp}
Cộng hai ma trận cùng cấp là cộng các phần tử tương ứng các vị trí với nhau:

Cho hai ma trận: \( A = (a_{ij})_{m \times n} \) và \( B = (b_{ij})_{m \times n} \). Ta có
\[
A + B = (a_{ij} + b_{ij})_{m \times n}
\]
\[
A + B = (a_{ij} + b_{ij})_{m \times n} \quad (1.4)
\]
\subsubsection*{Ví dụ 10.}
Cho hai ma trận:
\[
A = \begin{pmatrix}
1 & 2 & 3 \\
4 & 5 & 6
\end{pmatrix}
, \quad
B = \begin{pmatrix}
1 & -1 & 1 \\
-1 & 1 & -1
\end{pmatrix}
.
\]
Tính \( 2A, -4B, A + B, 2A - 4B \).

\textit{Giải}

Ta có
\[
2A = \begin{pmatrix}
2 & 4 & 6 \\
8 & 10 & 12
\end{pmatrix}
, \quad
-4B = \begin{pmatrix}
-4 & 4 & -4 \\
4 & -4 & 4
\end{pmatrix}
.
\]

và
\[
A + B = \begin{pmatrix}
2 & 1 & 4 \\
3 & 6 & 5
\end{pmatrix}
, \quad
2A - 4B = \begin{pmatrix}
6 & 0 & 10 \\
4 & 14 & 8
\end{pmatrix}
.
\]

\subsubsection*{1.4.3. Các tính chất}
Cho ba ma trận \( A, B, C \) cùng cấp và \( \forall \alpha, \beta \in \mathbb{R} \).
\begin{itemize}
    \item[a)] \( A + B = B + A \)
    \item[b)] \( (A + B) + C = A + (B + C) \)
    \item[c)] \( A + 0 = A \)
    \item[d)] \( A + (-A) = 0 \)
    \item[e)] \( 1 \cdot A = A \)
    \item[f)] \( (\alpha + \beta)A = \alpha A + \beta A \)
    \item[g)] \( \alpha(A + B) = \alpha A + \alpha B \)
    \item[h)] \( (\alpha \beta)A = \alpha(\beta A) = \beta(\alpha A) \)
\end{itemize}

\subsubsection*{1.4.4. Phép nhân hai ma trận}
Cho hai ma trận \( A = (a_{ij}) \in M_{m \times n} \), \( B = (b_{ij}) \in M_{n \times p} \). Ta định nghĩa ma trận tích của hai ma trận \( A, B \) là ma trận cấp \( m \times p \), ký hiệu \( AB = (c_{ij}) \in M_{m \times p} \), xác định bởi
\[
c_{ij} = a_{i1}b_{1j} + a_{i2}b_{2j} + \cdots + a_{in}b_{nj} = \sum_{k=1}^n a_{ik}b_{kj}, \quad \forall i = 1, m, \quad \forall j = 1, p \quad (1.5)
\]

\textit{Tính chất}

(i) \textit{Tính kết hợp}: Cho \( A \in M_{m \times n}, B \in M_{n \times p} \) và \( C \in M_{p \times q} \), ta có
\[
(AB)C = A(BC).
\]
\[ A(BC) = (AB)C. \]

(ii) \textit{Tính phân phối}: Với mọi ma trận \( A, B \in M_{m \times n} \) và \( C \in M_{n \times p} \), ta có
\[
(A + B)C = AC + BC,
\]
và với mọi ma trận \( C \in M_{m \times n} \) và \( A, B \in M_{n \times p} \), ta có
\[
C(A + B) = CA + CB.
\]

(iii) Với mọi ma trận \( A \in M_{m \times n}, B \in M_{n \times p} \) và với mọi \( k \in \mathbb{R} \), ta có
\[
k(AB) = (kA)B = A(kB).
\]

\textit{Hệ quả.} Cho \( A \) là ma trận vuông cấp \( n \). Ta có \( A^n = A \times A \times \cdots \times A \) (nhân \( n \) lần).

\subsubsection*{Ví dụ 11.}
Cho hai ma trận:
\[
A = \begin{pmatrix}
1 & 2 \\
-1 & 1 \\
2 & 3
\end{pmatrix} \in M_{3 \times 2}
, \quad
B = \begin{pmatrix}
2 & -3 & 4 \\
3 & 5 & 0
\end{pmatrix} \in M_{2 \times 3}
.
\]
Tính \( AB \) và \( (AB)^2 \).

\textit{Giải}

Ta có
\[
AB = \begin{pmatrix}
1 & 2 \\
-1 & 1 \\
2 & 3
\end{pmatrix}
\begin{pmatrix}
2 & -3 & 4 \\
3 & 5 & 0
\end{pmatrix}
= \begin{pmatrix}
8 & 7 & 4 \\
1 & 8 & -4 \\
13 & 9 & 8
\end{pmatrix}
.
\]

\[
(AB)^2 = (AB)(AB) = \begin{pmatrix}
8 & 7 & 4 \\
1 & 8 & -4 \\
13 & 9 & 8
\end{pmatrix}
\begin{pmatrix}
8 & 7 & 4 \\
1 & 8 & -4 \\
13 & 9 & 8
\end{pmatrix}
= \begin{pmatrix}
123 & 148 & 36 \\
-36 & 35 & -60 \\
217 & 235 & 123
\end{pmatrix}
.
\]

\subsubsection*{Ví dụ 12.}
Cho hai ma trận vuông cấp 4:
\[
A = \begin{pmatrix}
1 & 0 & 3 & 4 \\
-2 & 3 & 1 & -2 \\
2 & 4 & 3 & 3 \\
1 & -1 & 2 & 1
\end{pmatrix}
, \quad
B = \begin{pmatrix}
3 & 2 & -2 & 4 \\
2 & 1 & 1 & 3 \\
-1 & 0 & -3 & 0 \\
3 & 4 & 3 & 5
\end{pmatrix}
.
\]
Tính \( AB \) và \( BA \).
\textit{Giải}

Ta có
\[
AB = \begin{pmatrix}
1 & 0 & 3 & 4 \\
-2 & 3 & 1 & -2 \\
2 & 4 & 3 & 3 \\
1 & -1 & 2 & 1
\end{pmatrix}
\begin{pmatrix}
3 & 2 & -2 & 4 \\
2 & 1 & 1 & 3 \\
-1 & 0 & -3 & 0 \\
3 & 4 & 3 & 5
\end{pmatrix}
= \begin{pmatrix}
12 & 18 & 1 & 24 \\
-7 & -9 & -2 & -9 \\
18 & 20 & -7 & 33 \\
7 & 8 & 7 & 12
\end{pmatrix}
\]

\[
BA = \begin{pmatrix}
3 & 2 & -2 & 4 \\
2 & 1 & 1 & 3 \\
-1 & 0 & -3 & 0 \\
3 & 4 & 3 & 5
\end{pmatrix}
\begin{pmatrix}
1 & 0 & 3 & 4 \\
-2 & 3 & 1 & -2 \\
2 & 4 & 3 & 3 \\
1 & -1 & 2 & 1
\end{pmatrix}
= \begin{pmatrix}
-3 & 6 & -5 & 6 \\
6 & 8 & 5 & 12 \\
-10 & -6 & -15 & -13 \\
9 & 23 & 15 & 18
\end{pmatrix}
\]

\subsubsection*{1.5. Các phép biến đổi sơ cấp trên hàng}
\subsubsection*{1.5.1. Ba phép biến đổi sơ cấp trên hàng của ma trận}
i) Phép biến đổi loại 1: Đổi chỗ 2 hàng của ma trận.
\[
A \xrightarrow{(i) \leftrightarrow (i')} B
\]
ii) Phép biến đổi loại 2: Nhân một số thực khác không với một hàng.
\[
A \xrightarrow{(i)=\alpha(i) \\ \alpha \neq 0} B
\]
iii) Phép biến đổi loại 3: Thay 1 hàng bất kỳ bằng chính nó rồi cộng với một số thực nhân cho hàng khác.
\[
A \xrightarrow{(i)=(i)+\alpha(i')} B
\]

\textbf{Ví dụ 13.} Cho ma trận vuông cấp 3 như sau:
\[
A = \begin{pmatrix}
1 & 2 & 3 \\
2 & 2 & 4 \\
3 & 2 & 5
\end{pmatrix}
\]

\textit{Phép biến đổi loại 1:}
\[
\begin{pmatrix}
1 & 2 & 3 \\
2 & 2 & 4 \\
3 & 2 & 5
\end{pmatrix}
\xrightarrow{(2) \leftrightarrow (3)}
\begin{pmatrix}
1 & 2 & 3 \\
3 & 2 & 5 \\
2 & 2 & 4
\end{pmatrix}
\]

\textit{Phép biến đổi loại 2:}
\[
\begin{pmatrix}
1 & 2 & 3 \\
2 & 2 & 4 \\
3 & 2 & 5
\end{pmatrix}
\xrightarrow{(2) = \frac{1}{2}(2)}
\begin{pmatrix}
1 & 2 & 3 \\
1 & 1 & 2 \\
3 & 2 & 5
\end{pmatrix}
\]

\textit{Phép biến đổi loại 3:}
\[
\begin{pmatrix}
1 & 2 & 3 \\
2 & 2 & 4 \\
3 & 2 & 5
\end{pmatrix}
\xrightarrow{(2) = (2) - 2(1)}
\begin{pmatrix}
1 & 2 & 3 \\
0 & -2 & -2 \\
3 & 2 & 5
\end{pmatrix}
\]
\subsubsection*{1.5.2. Liên hệ giữa phép biến đổi sơ cấp trên hàng và phép nhân ma trận}
Cho ma trận \( A = (a_{ij})_{m \times n} \) và ma trận đơn vị cấp \( m \): \( I_m = \begin{pmatrix}
1 & 0 & \cdots & 0 \\
0 & 1 & \cdots & \vdots \\
\vdots & \vdots & \ddots & \vdots \\
0 & \cdots & 0 & 1
\end{pmatrix} \).

\textit{Định nghĩa:}

\[
I(i, j) = \begin{pmatrix}
1 & \cdots & \cdots & \cdots & 0 & 1 & \cdots & \cdots & \cdots & 0 \\
\vdots & \ddots & & & & & \ddots & & & \\
0 & \cdots & 1 & & & & & \ddots & & \\
1 & \cdots & \cdots & 0 & \cdots & & & & \ddots & \\
\vdots & & & & & & & & & \\
0 & & & & & 1 & & & & \ddots \\
\end{pmatrix} 
\]
\[
\text{dòng } i \quad \quad \quad \quad \quad \quad \quad \quad \quad \quad \quad \quad \quad \quad \quad \quad \quad \quad \text{dòng } j
\]

\[
I(i, \alpha) = \begin{pmatrix}
1 & \cdots & \cdots & \cdots & 0 & \alpha & \cdots & \cdots & \cdots & 0 \\
\vdots & \ddots & & & & & \ddots & & & \\
0 & \cdots & 1 & & & & & \ddots & & \\
\alpha & \cdots & \cdots & 0 & \cdots & & & & \ddots & \\
\vdots & & & & & & & & & \\
0 & & & & & \alpha & & & & \ddots \\
\end{pmatrix} 
\]
\[
\text{dòng } i \quad \quad \quad \quad \quad \quad \quad \quad \quad \quad \quad \quad \quad \quad \quad \quad \quad \quad \text{dòng } j
\]

\[
I(i, j, \alpha) = \begin{pmatrix}
1 & \cdots & \cdots & \cdots & 0 & 1 & \cdots & \cdots & \cdots & 0 \\
\vdots & \ddots & & & & & \ddots & & & \\
0 & \cdots & 1 & & & & & \ddots & & \\
1 & \cdots & \cdots & 0 & \alpha & & & & \ddots & \\
\vdots & & & & & & & & & \\
0 & & & & & 1 & & & & \ddots \\
\end{pmatrix} 
\]
\[
\text{dòng } i \quad \quad \quad \quad \quad \quad \quad \quad \quad \quad \quad \quad \quad \quad \quad \quad \quad \quad \text{dòng } j
\]

\textit{Lưu ý:}
\begin{itemize}
    \item [+] Phép hoán vị hai hàng của ma trận \( A \) được coi là thực hiện phép nhân ma trận \( I(i, j) \times A \).
    \item [+] Phép nhân một hàng của ma trận \( A \) với số thực \( \alpha \neq 0 \) được coi là phép nhân ma trận \( I(i, \alpha) \times A \).
    \item [+] Phép cộng vào hàng \( i \) hàng \( j \) đã nhân với \( \alpha \) (\( i \neq j \)) được coi là phép nhân ma trận \( I(i, j, \alpha) \times A \).
\end{itemize}
\subsection*{2. Định thức}
Xét ma trận vuông cấp \( n \): \( A = \begin{pmatrix}
a_{11} & a_{12} & \cdots & a_{1n} \\
a_{21} & a_{22} & \cdots & a_{2n} \\
\vdots & \vdots & \ddots & \vdots \\
a_{n1} & a_{n2} & \cdots & a_{nn}
\end{pmatrix} \)

Với mỗi số hạng \( a_{ij} \) (số hạng nằm ở hàng \( i \) và cột \( j \)), ma trận nhận được từ \( A \) bằng cách bỏ đi hàng thứ \( i \) và cột thứ \( j \) được gọi là \textit{ma trận bù} của \( A \) đối với số hạng \( a_{ij} \) ký hiệu là \( A_{ij} \).

\textbf{Ví dụ 14.} Cho ma trận vuông cấp 3:
\[
A = \begin{pmatrix}
1 & 4 & 7 \\
2 & 5 & 8 \\
3 & 6 & 9
\end{pmatrix}
\]
Ta có thể thành lập các ma trận bù cấp 2, chẳng hạn
\[
A_{11} = \begin{pmatrix}
5 & 8 \\
6 & 9
\end{pmatrix}
, \quad
A_{23} = \begin{pmatrix}
1 & 4 \\
3 & 6
\end{pmatrix}
, \quad
A_{33} = \begin{pmatrix}
1 & 4 \\
2 & 5
\end{pmatrix}
\]

\subsubsection*{2.1. Định nghĩa định thức ma trận vuông cấp n}
\textit{Định thức} của ma trận vuông \( A \in M_n \), ký hiệu \(\det(A)\) hay \(|A|\), là số thực được định nghĩa bằng quy nạp theo \( n \) như sau:
\begin{itemize}
    \item Với \( n = 1 \), nghĩa là \( A = (a_{11}) \), thì \(\det(A) = a_{11}\).
    \item Với \( n \ge 2 \), \( A = (a_{ij})_{n \times n} \), thì:
    \[
    \det(A) = (-1)^{1+1}a_{11}\det(A_{11}) + (-1)^{1+2}a_{12}\det(A_{12}) + \cdots + (-1)^{1+n}a_{1n}\det(A_{1n})
    \]
    \[
    \Leftrightarrow \det(A) = \sum_{j=1}^n (-1)^{1+j} a_{1j} \det(A_{1j}) \quad (1.6)
    \]
\end{itemize}

\textit{Xét một số trường hợp đặc biệt:}
\begin{itemize}
    \item Với \( n = 2 \), \( A = \begin{pmatrix}
    a_{11} & a_{12} \\
    a_{21} & a_{22}
    \end{pmatrix} \). Ta có
    \[
    \det(A) = a_{11}\det(A_{11}) + a_{12}\det(A_{12}) = a_{11}a_{22} - a_{21}a_{12}
    \]

    \item Với \( n = 3 \), ta có
    \[
    \det(A) = a_1 \begin{vmatrix}
    a_2 & a_3 \\
    b_2 & b_3
    \end{vmatrix} - a_2 \begin{vmatrix}
    a_1 & a_3 \\
    b_1 & b_3
    \end{vmatrix} + a_3 \begin{vmatrix}
    a_1 & a_2 \\
    b_1 & b_2
    \end{vmatrix}
    \]
    \[
    = a_1(b_2c_3 - b_3c_2) - a_2(b_1c_3 - b_3c_1) + a_3(b_1c_2 - b_2c_1)
    \]
    \[
    = a_1b_2c_3 + a_2b_3c_1 + a_3b_1c_2 - a_1b_3c_2 - a_2b_1c_3 - a_3b_2c_1.
    \]
\end{itemize}

\textit{Tính định thức của ma trận vuông cấp 3 bằng quy tắc 6 đường chéo (quy tắc Sarrus)}

Cho ma trận vuông cấp 3: \( A = \begin{pmatrix}
a_{11} & a_{12} & a_{13} \\
a_{21} & a_{22} & a_{23} \\
a_{31} & a_{32} & a_{33}
\end{pmatrix} \)

Xây dựng ma trận \( A'_{3 \times 3} \):
\[
A'_{3 \times 3} = \begin{pmatrix}
a_{11} & a_{12} & a_{13} & a_{11} & a_{12} \\
a_{21} & a_{22} & a_{23} & a_{21} & a_{22} \\
a_{31} & a_{32} & a_{33} & a_{31} & a_{32}
\end{pmatrix}
\]

\textit{Định thức của A}
\[
\det(A) = (a_{11}a_{22}a_{33} + a_{12}a_{23}a_{31} + a_{13}a_{21}a_{32}) - (a_{13}a_{22}a_{31} + a_{23}a_{32}a_{11} + a_{33}a_{21}a_{12})
\]

3 số hạng mang dấu cộng trong định thức là tích các phần tử nằm trên ba đường song song với đường chéo chính.

3 số hạng mang dấu âm trong định thức là tích các phần tử nằm trên ba đường song song với đường chéo phụ.

\textbf{Ví dụ 15.} Tính các định thức
\[
a) \begin{vmatrix}
1 & 4 \\
3 & 2
\end{vmatrix} = 1 \cdot 2 - 3 \cdot 2 = -2.
\]

\[
b) \begin{vmatrix}
2 & 3 & 4 \\
2 & 3 & 5 \\
1 & 5 & 2
\end{vmatrix} = 2 \begin{vmatrix}
3 & 5 \\
5 & 2
\end{vmatrix} - 3 \begin{vmatrix}
2 & 5 \\
1 & 2
\end{vmatrix} + 4 \begin{vmatrix}
2 & 3 \\
1 & 5
\end{vmatrix} = -1.
\]

\[
c) \begin{vmatrix}
1 & 2 & 3 \\
2 & 3 & 5 \\
5 & 4 & 3
\end{vmatrix} = 2 \cdot 2 \cdot 3 + 3 \cdot 5 \cdot 4 + 1 \cdot 4 \cdot 1 \cdot 4 - 5 \cdot 2 \cdot 4 - 4 \cdot 3 \cdot 3 - 5 \cdot 4 \cdot 3 \cdot 2 \cdot 1 \cdot 3 = -1.
\]

\[
d) \begin{vmatrix}
3 & -2 & 5 \\
-2 & 5 & -6 \\
3 & 4 & -6
\end{vmatrix} = (-1)^{1+1}3 \begin{vmatrix}
5 & -6 \\
4 & -6
\end{vmatrix} + (-1)^{1+2} (-2) \begin{vmatrix}
-2 & -6 \\
3 & -6
\end{vmatrix} + (-1)^{1+3} 5 \begin{vmatrix}
-2 & 5 \\
3 & 4
\end{vmatrix} = 31.
\]

\subsubsection*{2.2. Định lý khai triển định thức theo một hàng hay một cột bất kỳ}
Cho ma trận \( A = (a_{ij})_{n \times n}, 1 \le i_0, j_0 \le n \). Khi đó:
\[
\det(A) = \sum_{j=1}^n (-1)^{i_0 + j}a_{i_0 j} \det(A_{i_0 j}) \quad (1.7)
\]
\[
\det(A) = \sum_{i=1}^n (-1)^{i+j_0} a_{i j_0} \det(A_{i j_0}). \quad (1.8)
\]

Công thức (1.7) gọi là công thức khai triển theo hàng \(i_0\) và công thức (1.8) là công thức khai triển theo cột \(j_0\).

\subsubsection*{Ví dụ 16.}
\textit{Tính các định thức}

a) \( A = \begin{pmatrix}
1 & 2 & 3 & 4 \\
2 & 3 & 4 & 1 \\
3 & 4 & 1 & 2 \\
4 & 1 & 2 & 3
\end{pmatrix} \)

Tính định thức của ma trận \(A\). Chúng ta khai triển định thức này theo hàng 1:
\[
|A| = (-1)^{1+1} 1 \begin{vmatrix}
3 & 4 & 1 \\
4 & 1 & 2 \\
1 & 2 & 3
\end{vmatrix}
+ (-1)^{1+2} 2 \begin{vmatrix}
2 & 4 & 1 \\
3 & 1 & 2 \\
4 & 2 & 3
\end{vmatrix}
\]
\[
+ (-1)^{1+3} 3 \begin{vmatrix}
2 & 3 & 4 \\
3 & 4 & 1 \\
4 & 1 & 2
\end{vmatrix}
+ (-1)^{1+4} 4 \begin{vmatrix}
2 & 3 & 4 \\
3 & 4 & 1 \\
4 & 1 & 2
\end{vmatrix}
= -36 + 8 + 12 + 176 = 160.
\]

b) \( B = \begin{pmatrix}
1 & 3 & 0 & a \\
2 & b & 0 & 0 \\
3 & 4 & c & 5 \\
d & 0 & 0 & 0
\end{pmatrix} \)

Tính định thức của ma trận \(B\). Chúng ta khai triển định thức này theo hàng 4:
\[
|B| = (-1)^{4+1} d \begin{vmatrix}
3 & 0 & a \\
b & 0 & 0 \\
4 & c & 5
\end{vmatrix}
= -d(-1)^{3+2} c \begin{vmatrix}
3 & a \\
b & 0
\end{vmatrix}
= -abcd.
\]

c) \( C = \begin{pmatrix}
1 & -1 & -2 & 0 \\
2 & m & 0 & 2 \\
3 & 0 & 4 & 1 \\
4 & 2 & 1 & 0
\end{pmatrix} \)

Tính định thức của ma trận \(C\). Chúng ta khai triển định thức này theo cột 4:
\[
|C| = (-1)^{2+4} 2 \begin{vmatrix}
1 & -1 & -2 \\
2 & m & 0 \\
3 & 0 & 4 \\
4 & 2 & 1
\end{vmatrix}
+ (-1)^{3+4} 3 \begin{vmatrix}
1 & -1 & -2 \\
2 & m & 0 \\
4 & 2 & 1
\end{vmatrix}
= -48 - 9m.
\]
\subsubsection*{2.3. Các tính chất định thức}

i) \textit{Tính chất 1.} Cho ma trận vuông \(A\). Ta có \(\det(A) = \det(A^T)\).

\textbf{Ví dụ 17.} Cho ma trận:
\[
A = \begin{pmatrix}
1 & 3 & 4 \\
2 & 5 & 1 \\
3 & 1 & 2
\end{pmatrix}
\]
Ta có: \(\det(A) = \det(A^T) = -46\).

ii) \textit{Tính chất 2.} Cho \(A, B\) là hai ma trận vuông. Ta có
\[
\det(AB) = \det(BA) = \det(A)\det(B).
\]

\textbf{Ví dụ 18.} Cho hai ma trận: \( A = \begin{pmatrix}
1 & 2 \\
3 & 4
\end{pmatrix}, B = \begin{pmatrix}
2 & 1 \\
4 & 3
\end{pmatrix} \)
\begin{itemize}
    \item[a)] Tính \( AB \) và \( BA \).
    \item[b)] Tính \(\det(A)\), \(\det(B)\), \(\det(AB)\), \(\det(BA)\).
\end{itemize}

\textit{Giải}
\begin{itemize}
    \item[a)] Ta có: \( AB = \begin{pmatrix}
    10 & 7 \\
    22 & 15
    \end{pmatrix}, BA = \begin{pmatrix}
    5 & 8 \\
    13 & 20
    \end{pmatrix} \)
    \item[b)] Ta có: \(\det(A) = -2; \det(B) = 2; \det(AB) = \det(BA) = -4\).
\end{itemize}

iii) \textit{Tính chất 3.} Cho \(I\) là ma trận đơn vị cấp \(n\). Ta có \(\det(I) = 1\).

iv) \textit{Tính chất 4.} Cho ba ma trận \(A, B, C \in M_n\) thỏa mãn:
\[
[C]_{ij} = [A]_{ij} + [B]_{ij} \quad \text{và} \quad [C]_{ij} = [A]_{ij} = [B]_{ij}, \quad i = 2, 3, \ldots, n; j = 1, 2, \ldots, n.
\]
Ta có: \(\det(C) = \det(A) + \det(B)\).

\textbf{Ví dụ 19.} Cho ba ma trận
\[
A = \begin{pmatrix}
a & b & c \\
1 & 2 & 3 \\
2 & 3 & 4
\end{pmatrix},
B = \begin{pmatrix}
b & c & a \\
1 & 2 & 3 \\
2 & 3 & 4
\end{pmatrix},
C = \begin{pmatrix}
a + b & b + c & c + a \\
1 & 2 & 3 \\
2 & 3 & 4
\end{pmatrix}.
\]
\textit{Chứng minh rằng:} \(\det(C) = \det(A) + \det(B)\).

\textit{Giải}
\[
\det(A) = -a + 2b - c; \det(B) = -a - b + 2c; \det(C) = -2a + b + c.
\]
Vậy \(\det(C) = \det(A) + \det(B)\).

v) \textit{Tính chất 5.} Cho số thực \(k \in \mathbb{R}\) và ma trận \(A \in M_n\). Ta có \(\det(kA) = k^n \det(A)\).

\textbf{Ví dụ 20.} Cho ma trận \(A = \begin{pmatrix}
1 & 3 & 4 \\
2 & 5 & 1 \\
3 & 1 & 2
\end{pmatrix} \).
Ta có:
\[
\det(A) = -46 \Rightarrow \det(2A) = 2^3 \det(A) = -368.
\]

\subsubsection*{2.4. Định lý sự thay đổi của định thức qua các phép biến đổi}
i) Nếu \(A \xrightarrow{(i) \leftrightarrow (i')} B\) thì \(\det(B) = -\det(A)\).\\
ii) Nếu \(A \xrightarrow{(i)=\alpha(i) \\ \alpha \neq 0} B\) thì \(\det(B) = \alpha \det(A)\).\\
iii) Định thức của ma trận có 2 dòng hoặc hai cột tỉ lệ với nhau thì bằng 0.\\
iv) Nếu \(A \xrightarrow{(i)=(i)+\alpha(i')} B\) thì \(\det(B) = \det(A)\).\\
v) Định thức của ma trận tam giác trên bằng tích các số hạng nằm trên đường.

\textbf{Ví dụ 21.} Cho ma trận: \(A = \begin{pmatrix}
1 & 2 & 3 \\
2 & 3 & 1 \\
3 & 1 & 5
\end{pmatrix} \)

a) Thực hiện phép biến đổi loại 1.
\[
A = \begin{pmatrix}
1 & 2 & 3 \\
2 & 3 & 1 \\
3 & 1 & 5
\end{pmatrix} \xrightarrow{(1) \leftrightarrow (2)} B = \begin{pmatrix}
2 & 3 & 1 \\
1 & 2 & 3 \\
3 & 1 & 5
\end{pmatrix}.
\]
Ta có: \(\det(A) = -21; \det(B) = 21; \det(B) = -\det(A)\)

b) Thực hiện phép biến đổi loại 2
\[
A = \begin{pmatrix}
1 & 2 & 3 \\
2 & 3 & 1 \\
3 & 1 & 5
\end{pmatrix} \xrightarrow{(1)=2(1)} B = \begin{pmatrix}
2 & 4 & 6 \\
2 & 3 & 1 \\
3 & 1 & 5
\end{pmatrix}.
\]
Ta có: \(\det(A) = -21; \det(B) = -42 = 2\det(A)\).

c) Thực hiện phép biến đổi loại 3
\[
A = \begin{pmatrix}
1 & 2 & 3 \\
2 & 3 & 1 \\
3 & 1 & 5
\end{pmatrix} \xrightarrow{(2)=(2)-2(1)} B = \begin{pmatrix}
1 & 2 & 3 \\
0 & -1 & -5 \\
3 & 1 & 5
\end{pmatrix}.
\]
Ta có: \(\det(A) = \det(B) = -21\).
\subsubsection*{2.3. Các tính chất định thức}


i) \textit{Tính chất 1.} Cho ma trận vuông \(A\). Ta có \(\det(A) = \det(A^T)\).

\textbf{Ví dụ 17.} Cho ma trận:
\[
A = \begin{pmatrix}
1 & 3 & 4 \\
2 & 5 & 1 \\
3 & 1 & 2
\end{pmatrix}
\]
Ta có: \(\det(A) = \det(A^T) = -46\).\\
ii) \textit{Tính chất 2.} Cho \(A, B\) là hai ma trận vuông. Ta có
\[
\det(AB) = \det(BA) = \det(A)\det(B).
\]

\textbf{Ví dụ 18.} Cho hai ma trận: \( A = \begin{pmatrix}
1 & 2 \\
3 & 4
\end{pmatrix}, B = \begin{pmatrix}
2 & 1 \\
4 & 3
\end{pmatrix} \)
\begin{itemize}
    \item[a)] Tính \( AB \) và \( BA \).
    \item[b)] Tính \(\det(A)\), \(\det(B)\), \(\det(AB)\), \(\det(BA)\).
\end{itemize}

\textit{Giải}
\begin{itemize}
    \item[a)] Ta có: \( AB = \begin{pmatrix}
    10 & 7 \\
    22 & 15
    \end{pmatrix}, BA = \begin{pmatrix}
    5 & 8 \\
    13 & 20
    \end{pmatrix} \)
    \item[b)] Ta có: \(\det(A) = -2; \det(B) = 2; \det(AB) = \det(BA) = -4\).
\end{itemize}
iii) \textit{Tính chất 3.} Cho \(I\) là ma trận đơn vị cấp \(n\). Ta có \(\det(I) = 1\).\\
iv) \textit{Tính chất 4.} Cho ba ma trận \(A, B, C \in M_n\) thỏa mãn:
\[
[C]_{ij} = [A]_{ij} + [B]_{ij} \quad \text{và} \quad [C]_{ij} = [A]_{ij} = [B]_{ij}, \quad i = 2, 3, \ldots, n; j = 1, 2, \ldots, n.
\]
Ta có: \(\det(C) = \det(A) + \det(B)\).

\textbf{Ví dụ 19.} Cho ba ma trận
\[
A = \begin{pmatrix}
a & b & c \\
1 & 2 & 3 \\
2 & 3 & 4
\end{pmatrix},
B = \begin{pmatrix}
b & c & a \\
1 & 2 & 3 \\
2 & 3 & 4
\end{pmatrix},
C = \begin{pmatrix}
a + b & b + c & c + a \\
1 & 2 & 3 \\
2 & 3 & 4
\end{pmatrix}.
\]
\textit{Chứng minh rằng:} \(\det(C) = \det(A) + \det(B)\).

\textit{Giải}
\[
\det(A) = -a + 2b - c; \det(B) = -a - b + 2c; \det(C) = -2a + b + c.
\]
Vậy \(\det(C) = \det(A) + \det(B)\).\\
v) \textit{Tính chất 5.} Cho số thực \(k \in \mathbb{R}\) và ma trận \(A \in M_n\). Ta có \(\det(kA) = k^n \det(A)\).

\textbf{Ví dụ 20.} Cho ma trận \(A = \begin{pmatrix}
1 & 3 & 4 \\
2 & 5 & 1 \\
3 & 1 & 2
\end{pmatrix} \).
Ta có:
\[
\det(A) = -46 \Rightarrow \det(2A) = 2^3 \det(A) = -368.
\]

\subsubsection*{1.2.4. Định lý sự thay đổi của định thức qua các phép biến đổi}

i) Nếu \(A \xrightarrow{(i) \leftrightarrow (i')} B\) thì \(\det(B) = -\det(A)\).\\
ii) Nếu \(A \xrightarrow{(i)=\alpha(i) \\ \alpha \neq 0} B\) thì \(\det(B) = \alpha \det(A)\).\\
iii) Định thức của ma trận có 2 dòng hoặc hai cột tỉ lệ với nhau thì bằng 0.\\
iv) Nếu \(A \xrightarrow{(i)=(i)+\alpha(i')} B\) thì \(\det(B) = \det(A)\).\\
v) Định thức của ma trận tam giác trên bằng tích các số hạng nằm trên đường.

\textbf{Ví dụ 21.} Cho ma trận: \(A = \begin{pmatrix}
1 & 2 & 3 \\
2 & 3 & 1 \\
3 & 1 & 5
\end{pmatrix} \)

a) Thực hiện phép biến đổi loại 1.
\[
A = \begin{pmatrix}
1 & 2 & 3 \\
2 & 3 & 1 \\
3 & 1 & 5
\end{pmatrix} \xrightarrow{(1) \leftrightarrow (2)} B = \begin{pmatrix}
2 & 3 & 1 \\
1 & 2 & 3 \\
3 & 1 & 5
\end{pmatrix}.
\]
Ta có: \(\det(A) = -21; \det(B) = 21; \det(B) = -\det(A)\)

b) Thực hiện phép biến đổi loại 2
\[
A = \begin{pmatrix}
1 & 2 & 3 \\
2 & 3 & 1 \\
3 & 1 & 5
\end{pmatrix} \xrightarrow{(1)=2(1)} B = \begin{pmatrix}
2 & 4 & 6 \\
2 & 3 & 1 \\
3 & 1 & 5
\end{pmatrix}.
\]
Ta có: \(\det(A) = -21; \det(B) = -42 = 2\det(A)\).

c) Thực hiện phép biến đổi loại 3
\[
A = \begin{pmatrix}
1 & 2 & 3 \\
2 & 3 & 1 \\
3 & 1 & 5
\end{pmatrix} \xrightarrow{(2)=(2)-2(1)} B = \begin{pmatrix}
1 & 2 & 3 \\
0 & -1 & -5 \\
3 & 1 & 5
\end{pmatrix}.
\]
Ta có: \(\det(A) = \det(B) = -21\).

\subsection*{1.3. Ma trận nghịch đảo}

\subsubsection*{3.1. Định nghĩa ma trận nghịch đảo}
Cho \( A, B \in M_n \), ta nói \( A, B \) là hai \textit{ma trận nghịch đảo} của nhau nếu \( AB = BA = I_n \).

Khi đó, ta nói \( A \) và \( B \) là các \textit{ma trận khả nghịch}.

Ký hiệu \( B = A^{-1} \) hay \( A = B^{-1} \).

\textit{Tính chất:} Ma trận \( A \in M_n \) khả nghịch khi và chỉ khi \(\det(A) \neq 0\).

\textbf{Ví dụ 23.} Định \( m \) để ma trận sau khả nghịch
\[
A = \begin{pmatrix}
1 & -2 & m \\
3 & m & 4 \\
2 & 3 & m
\end{pmatrix}
\]

\textit{Giải}

Từ ma trận \( A \) ta biến đổi như sau
\[
A = \begin{pmatrix}
1 & -2 & m \\
3 & m & 4 \\
2 & 3 & m
\end{pmatrix}
\xrightarrow{(2)=(2)-3(1) \\ (3)=(3)-2(1)}
\begin{pmatrix}
1 & -2 & m \\
0 & m+6 & 4-3m \\
0 & 7 & m-2
\end{pmatrix}
\]

Ta có
\[
\det(A) = (-1)^{1+1} 1 \begin{vmatrix}
m+6 & 4-3m \\
7 & m-2
\end{vmatrix} = -m^2 + 15m - 28
\]

Ma trận khả nghịch khi và chỉ khi \(\det(A) \neq 0 \Leftrightarrow m \neq \frac{15 \pm \sqrt{113}}{2} \).

\textbf{Ví dụ 24.} Cho ma trận \( A \in M_n \) thỏa mãn \( A^2 - 2A + I_n = 0 \). Chứng minh rằng ma trận \( A \) khả nghịch.

\textit{Giải}

Từ đẳng thức \( A^2 - 2A + I_n = 0 \), ta có \( I_n = A(2I_n - A) \) (*)

Lấy định thức hai vế của (*), ta có
\[
1 = \det(I_n) = \det[A(2I_n - A)] = \det(A)\det(2I_n - A)
\]

Suy ra \(\det(A) \neq 0\). Vậy \( A \) khả nghịch.

\subsubsection*{3.2. Giải thuật tìm ma trận nghịch đảo}

\textit{Phương pháp 1.} Tìm \( A^{-1} \) bằng định thức.
\begin{itemize}
    \item[+] Bước 1. Cho \( A \in M_n \), \(\det(A) \neq 0\).
    \item[+] Bước 2. Tính các phần bù đại số của \(A\) đối với phần tử \(a_{ij}\) (\(A_{ij}^* = (-1)^{i+j} \left|A_{ij}\right|\)).
    \item[+] Bước 3. Đặt \(A^* = (A_{ji}^*)_n \in M_n\). Khi đó:
    \[
    A^{-1} = \frac{1}{\det A} A^* \quad (1.9)
    \]
\end{itemize}
    
\textit{Phương pháp 2.} Dùng phép biến đổi sơ cấp theo hàng.
\begin{itemize}
    \item[+] Bước 1. Lập ma trận \((A \mid I_n)\) là ma trận gồm \(n\) hàng và \(2n\) cột, trong đó \(n\) cột đầu của \((A \mid I_n)\) chính là ma trận \(A\), \(n\) cột cuối của \((A \mid I_n)\) là ma trận đơn vị \(I_n\).
    \item[+] Bước 2. Bằng các phép biến đổi sơ cấp theo hàng, ta có thể chuyển ma trận \((A \mid I_n)\) về ma trận \((I_n \mid B)\) và khi đó \(B = A^{-1}\).

    Nếu ma trận \(A\) không chuyển được về ma trận đơn vị thì ma trận \(A\) không khả nghịch.
\end{itemize}

\textbf{Ví dụ 25.} Tìm ma trận nghịch đảo của ma trận sau bằng phương pháp định thức
\[
A = \begin{pmatrix}
1 & -1 & 1 \\
-1 & 2 & 1 \\
-2 & 3 & 1
\end{pmatrix}
\]

\textit{Giải}
Ta có \(\det(A) = 1\), do đó \(A\) khả nghịch và \(A^{-1}\) được tính bởi công thức sau
\[
A^{-1} = \frac{1}{\det A} A^* = \frac{1}{1} A^* = A^*,
\]
với \(A_{ij}^* = (-1)^{i+j} \left|A_{ij}\right|\),
\[
A_{11}^* = (-1)^{1+1} \left| \begin{pmatrix}
2 & 1 \\
3 & 1
\end{pmatrix} \right| = 1, \quad A_{12}^* = (-1)^{1+2} \left| \begin{pmatrix}
-1 & 1 \\
-2 & 1
\end{pmatrix} \right| = 1,
\]

\[
A_{13}^* = (-1)^{1+3} \left| \begin{pmatrix}
-1 & 2 \\
-2 & 3
\end{pmatrix} \right| = 1, \quad A_{21}^* = (-1)^{2+1} \left| \begin{pmatrix}
-1 & 1 \\
3 & 1
\end{pmatrix} \right| = 4,
\]
\[ A_{22}^* = (-1)^{2+2} \left| \begin{pmatrix}
1 & 1 \\
-2 & 1
\end{pmatrix} \right| = 3, \quad A_{23}^* = (-1)^{2+3} \left| \begin{pmatrix}
1 & -1 \\
-2 & 3
\end{pmatrix} \right| = -1,
\]

\[
A_{31}^* = (-1)^{3+1} \left| \begin{pmatrix}
-1 & 1 \\
2 & 1
\end{pmatrix} \right| = -3, 
\]
\[
A_{32}^* = (-1)^{3+2} \left| \begin{pmatrix}
1 & 1 \\
-1 & 1
\end{pmatrix} \right| = -2,
\]
\[
A_{33}^* = (-1)^{3+3} \left| \begin{pmatrix}
1 & -1 \\
-1 & 2
\end{pmatrix} \right| = 1.
\]

Vậy ma trận nghịch đảo của ma trận \(A\) là
\[
A^{-1} = \frac{1}{\det(A)} A^* = \begin{pmatrix}
-1 & 4 & -3 \\
-1 & 3 & -2 \\
1 & -1 & 1
\end{pmatrix}.
\]

\textbf{Ví dụ 26.} Tìm ma trận nghịch đảo sau bằng phương pháp biến đổi sơ cấp trên hàng
\[
A = \begin{pmatrix}
1 & -1 & 1 \\
-1 & 2 & 1 \\
-2 & 3 & 1
\end{pmatrix}
\]

\textit{Giải}

Thực hiện các phép biến đổi sơ cấp trên dòng như sau
\[
\left( A \mid I_3 \right) = \left( \begin{array}{ccc|ccc}
1 & -1 & 1 & 1 & 0 & 0 \\
-1 & 2 & 1 & 0 & 1 & 0 \\
-2 & 3 & 1 & 0 & 0 & 1 \\
\end{array} \right)
\xrightarrow{(2) := (2) + (1) \\ (3) := (3) + 2(1)}
\left( \begin{array}{ccc|ccc}
1 & -1 & 1 & 1 & 0 & 0 \\
0 & 1 & 2 & 1 & 1 & 0 \\
0 & 1 & 3 & 1 & 0 & 1 \\
\end{array} \right)
\]
\[
\xrightarrow{(3) := (3) - (2)}
\left( \begin{array}{ccc|ccc}
1 & -1 & 1 & 1 & 0 & 0 \\
0 & 1 & 2 & 1 & 1 & 0 \\
0 & 0 & 1 & 0 & -1 & 1 \\
\end{array} \right)
\xrightarrow{(1) := (1) + (2)}
\left( \begin{array}{ccc|ccc}
1 & 0 & 3 & 2 & 1 & 0 \\
0 & 1 & 2 & 1 & 1 & 0 \\
0 & 0 & 1 & 0 & -1 & 1 \\
\end{array} \right)
\]
\[
\xrightarrow{(1) := (1) - 3(3) \\ (2) := (2) - 2(3)}
\left( \begin{array}{ccc|ccc}
1 & 0 & 0 & 2 & 4 & -3 \\
0 & 1 & 0 & 1 & 3 & -2 \\
0 & 0 & 1 & 0 & -1 & 1 \\
\end{array} \right) = (I_3 \mid A^{-1})
\]

Vậy ma trận nghịch đảo của \(A\) là
\[
A^{-1} = \begin{pmatrix}
-1 & 4 & -3 \\
-1 & 3 & -2 \\
1 & -1 & 1
\end{pmatrix}.
\]

\subsubsection*{1.3.3. Định lý sự tồn tại của ma trận nghịch đảo}

Nếu ma trận \(A\) khả nghịch thì ma trận nghịch đảo \(A^{-1}\) tồn tại duy nhất.

\subsubsection*{1.3.4. Một số tính chất của ma trận nghịch đảo}

Nếu \(A, B\) là những ma trận vuông cấp \(n\) khả nghịch thì
\begin{itemize}
    \item[i)] \((A^{-1})^{-1} = A\),
    \item[ii)] \((AB)^{-1} = B^{-1}A^{-1}\),
    \item[iii)] \((A^T)^{-1} = (A^{-1})^T\),
    \item[iv)] \((\alpha A)^{-1} = \frac{1}{\alpha} A^{-1} \quad \text{với} \quad \alpha \neq 0.\)
\end{itemize}

\textbf{Ví dụ 27.} Giải phương trình ma trận \(XA = B\) với
\[
A = \begin{pmatrix}
1 & -1 & 1 \\
-1 & 2 & 1 \\
-2 & 3 & 1
\end{pmatrix}
\quad \text{và} \quad
B = \begin{pmatrix}
1 & -2 & 0 \\
4 & 5 & 1 \\
0 & 1 & 3
\end{pmatrix}
\]

\textit{Giải}

Theo ví dụ 26, ta có
\[
A = \begin{pmatrix}
1 & -1 & 1 \\
-1 & 2 & 1 \\
-2 & 3 & 1
\end{pmatrix}
\quad \text{và} \quad
B = \begin{pmatrix}
1 & -2 & 0 \\
4 & 5 & 1 \\
0 & 1 & 3
\end{pmatrix}.
\]

Từ phương trình ma trận nhân bên phải hai vế cho \(A^{-1}\), ta được
\[
X = BA^{-1} = \begin{pmatrix}
1 & -2 & 0 \\
4 & 5 & 1 \\
0 & 1 & 3
\end{pmatrix}
\begin{pmatrix}
-1 & 4 & -3 \\
-1 & 3 & -2 \\
1 & -1 & 1
\end{pmatrix} = \begin{pmatrix}
-8 & 30 & -21 \\
11 & -5 & 2 \\
2 & 0 & 1
\end{pmatrix}.
\]

\subsection*{4. Hạng của ma trận}

\subsubsection*{4.1. Định nghĩa tổng quát hạng của một ma trận}
Cho ma trận \(A \in M_{m \times n}\), ta gọi hạng của ma trận \(A\) bằng \(r\) nếu
\begin{itemize}
    \item[i)] Mọi định thức con của \(A\) cấp lớn hơn \(r\) đều bằng 0.
    \item[ii)] Trong \(A\) tồn tại một định thức con cấp \(r\) khác 0.
\end{itemize}

Ta ký hiệu hạng của ma trận \(A\) là \(\text{rank}(A)\) hay vắn tắt là \(r(A)\). Khi \(A\) là ma trận 0, ta quy ước \(r(A) = 0\).

\textit{Lưu ý rằng:} \(0 \le r(A) \le \min\{m,n\}\).

\subsubsection*{4.2. Tính chất}
\begin{itemize}
    \item[i)] Hạng của ma trận không thay đổi qua các phép biến đổi sơ cấp trên hàng, nghĩa là nếu \(B\) là ma trận nhận được từ \(A\) sau hữu hạn các phép biến đổi sơ cấp thì \(r(A) = r(B)\).
    \item[ii)] Hạng của ma trận không thay đổi qua phép chuyển vị, nghĩa là \(r(A) = r(A^T)\).
    \item[iii)] Nếu \(A\) là ma trận bậc thang theo hàng thì hạng của \(A\) bằng số hàng khác không của nó.
\end{itemize}

\subsubsection*{4.3. Phương pháp tìm hạng của ma trận}
Dùng các phép biến đổi sơ cấp trên hàng. Dùng các phép biến đổi sơ cấp trên hàng để đưa ma trận \(A\) về dạng ma trận bậc thang theo hàng \(B\). Khi đó, \(r(A)\) bằng số hàng khác không của ma trận \(B\).

\textbf{Ví dụ 28.} Cho ma trận:
\textbf{Tìm hạng của ma trận \( A \)}

\textit{Giải.}
Biến ma trận \( A \) về ma trận bậc thang theo hàng
\[
A = \begin{pmatrix}
1 & 2 & -1 & 0 \\
-1 & 2 & 4 & 2 \\
3 & 6 & -3 & 0
\end{pmatrix}
\xrightarrow{(2) = (2) + (1) \\ (3) = (3) - 3(1)}
\begin{pmatrix}
1 & 2 & -1 & 0 \\
0 & 4 & 3 & 2 \\
0 & 0 & 0 & 0
\end{pmatrix} = B
\]

Ma trận \( B \) là ma trận bậc thang có hai dòng khác dòng không nên \( r(A) = r(B) = 2 \).

\textbf{Ví dụ 29.} Biện luận theo hạng của ma trận sau:
\[
A = \begin{pmatrix}
1 & 3 & 2 \\
2 & 1 & m \\
3 & 8 & 5
\end{pmatrix}
\]

\textit{Giải}

Biến ma trận \( A \) về ma trận bậc thang theo dòng (hoặc ma trận tam giác trên)
\[
A = \begin{pmatrix}
1 & 3 & 2 \\
2 & 1 & m \\
3 & 8 & 5
\end{pmatrix}
\xrightarrow{(2) = (2) - 2(1) \\ (3) = (3) - 3(1)}
\begin{pmatrix}
1 & 3 & 2 \\
0 & -5 & m - 4 \\
0 & -1 & -1
\end{pmatrix}
\]
\[
\xrightarrow{(3) = (3) - \frac{1}{5}(2)}
\begin{pmatrix}
1 & 3 & 2 \\
0 & -5 & m - 4 \\
0 & 0 & -1 - m
\end{pmatrix} = B
\]

Ma trận \( B \) là ma trận bậc thang theo dòng, ta có \(\text{rank}(A) = \text{rank}(B)\).

\textit{Biện luận}
\begin{itemize}
    \item Nếu \( m = -1 \) thì \(\text{rank}(A) = 2\).
    \item Nếu \( m \neq -1 \) thì \(\text{rank}(A) = 3\).
\end{itemize}

\subsubsection*{4.4. Một số bất đẳng thức về hạng của ma trận}

a) Cho \( A \) và \( B \) là hai ma trận vuông cấp \( n \). Khi đó
\begin{itemize}
    \item[i)] \( r(A + B) \le r(A) + r(B) \).
    \item[ii)] \( r(A) + r(B) - n \le r(AB) \le \min\{r(A), r(B)\} \).
    \item[iii)] Nếu ma trận \( B \) khả nghịch thì \( r(AB) = r(BA) = r(A) \).
\end{itemize}
b) Cho ma trận \( A \in M_{m \times n} \) và ma trận \( B \in M_{n \times p} \). Khi đó \( r(AB) \le \min\{r(A), r(B)\} \).

\textbf{Ví dụ 30.} Cho \( A \) là ma trận cấp \( 3 \times 2 \), \( B \) là ma trận cấp \( 2 \times 3 \) sao cho
\[
AB = \begin{pmatrix}
8 & 2 & -2 \\
2 & 5 & 4 \\
-2 & 4 & 5
\end{pmatrix}
\]

a) Tìm hạng của ma trận \( AB \).

b) Chứng minh ma trận \( BA \) khả nghịch và tìm \( BA \).

\textit{Giải}

a) Tìm hạng của ma trận \( AB \).

Thực hiện phép biến đổi sơ cấp trên hàng của ma trận \( AB \), biến ma trận \( AB \) về ma trận bậc thang như sau.
\[
AB = \begin{pmatrix}
8 & 2 & -2 \\
2 & 5 & 4 \\
-2 & 4 & 5
\end{pmatrix}
\xrightarrow{(2) = (2) - 4(1) \\ (3) = (3) + 2(1)}
\begin{pmatrix}
8 & 2 & -2 \\
0 & 18 & 18 \\
0 & 18 & 18
\end{pmatrix}
\xrightarrow{(3) = (3) - (2)}
\begin{pmatrix}
8 & 2 & -2 \\
0 & 18 & 18 \\
0 & 0 & 0
\end{pmatrix}
\]

Vậy hạng của ma trận \( AB \) là \( r(AB) = 2 \).

b) Chứng minh ma trận \( BA \) khả nghịch và tìm \( BA \).

Ta có
\[
(AB)^2 = \begin{pmatrix}
72 & 18 & -18 \\
18 & 45 & 36 \\
-18 & 36 & 45
\end{pmatrix} = 9 \begin{pmatrix}
8 & 2 & -2 \\
2 & 5 & 4 \\
-2 & 4 & 5
\end{pmatrix} = 9AB
\]

Ta có: \( 2 = r(AB) = r[(AB)^2] = r[(A(BA)B)] \le r(BA) \le 2 \)

Vậy \( r(BA) = 2 \) nên \( BA \) là ma trận khả nghịch.

Ta có
\[
(BA)^3 = (BA)(BA)(BA) = B(AB)^2 A = 9B(AB)A = 9(BA)^2
\]

Nhân hai vế của đẳng thức cho \( (BA)^{-1} \) hai lần, ta được
\[
BA = 9I_2 = \begin{pmatrix}
9 & 0 \\
0 & 9
\end{pmatrix}.
\]
\subsection*{5. Bài tập}
\textbf{Bài số 1.} Thực hiện các phép tính trên các ma trận sau:
1. Tính \( 5A - 3B + 2C \), biết:
\[
A = \begin{pmatrix}
1 & 2 \\
-1 & 0 \\
2 & 1
\end{pmatrix},
B = \begin{pmatrix}
1 & 3 \\
2 & 1 \\
-3 & -2
\end{pmatrix},
C = \begin{pmatrix}
2 & 5 \\
0 & 3 \\
4 & 2
\end{pmatrix}.
\]

2. Tính \( AB, BA \) biết:
\[
A = \begin{pmatrix}
2 & -1 \\
1 & 0 \\
-3 & 4
\end{pmatrix}
\quad \text{và} \quad
B = \begin{pmatrix}
1 & -2 & 5 \\
3 & 4 & 0
\end{pmatrix}.
\]

3. Tính \( AB, BA \) biết:
\[
A = \begin{pmatrix}
1 & -3 & 2 \\
3 & -4 & 1 \\
2 & -5 & 3
\end{pmatrix}
\quad \text{và} \quad
B = \begin{pmatrix}
2 & 5 & 6 \\
1 & 2 & 5 \\
1 & 3 & 2
\end{pmatrix}.
\]

\textit{Đáp số:}
1) \( 5A - 3B + 2C = \begin{pmatrix}
6 & 11 \\
-11 & 3 \\
27 & 15
\end{pmatrix} \)

2) \( AB = \begin{pmatrix}
-1 & -8 & 10 \\
1 & -2 & 5 \\
9 & 22 & -15
\end{pmatrix},
BA = \begin{pmatrix}
-15 & 19 \\
10 & -3
\end{pmatrix} \)

3) \( AB = \begin{pmatrix}
1 & 5 & -5 \\
3 & 10 & 0 \\
2 & 9 & -7
\end{pmatrix},
BA = \begin{pmatrix}
29 & -56 & 27 \\
17 & -36 & 19 \\
14 & -25 & 11
\end{pmatrix} \)\\
\textbf{Bài số 2.} Cho \( A = \begin{pmatrix}
2 & 0 & 1 \\
3 & 1 & 2 \\
0 & -1 & 0
\end{pmatrix} \). Tính \( f(A) \) với \( f(x) = x^2 - 5x + 3 \).

\textit{Đáp số:} \( f(A) = \begin{pmatrix}
-3 & -1 & -3 \\
-6 & -3 & -5 \\
-3 & 4 & 1
\end{pmatrix} \).\\
\textbf{Bài số 3.} Cho các ma trận: \( A = \begin{pmatrix}
2 & -1 & 3 \\
0 & 1 & 2
\end{pmatrix} \), \( B = \begin{pmatrix}
-2 & 1 \\
2 & 0 \\
1 & -1
\end{pmatrix} \), \( C = \begin{pmatrix}
1 & 1 \\
0 & 1
\end{pmatrix} \).
\begin{enumerate}
    \item Có thể thành lập được tích của các ma trận nào trong các ma trận trên.
    \item Tính \( AB, ABC \).
    \item Tính \((AB)^3\), \(C^n\) với \(n \in \mathbb{N}\).

    \item Tìm ma trận chuyển vị của \(A\) và tính \(A^T C\)
    
\end{enumerate}

\textit{Đáp số:}
1) \(AB, BA, BC, CA\);
2) \(AB = \begin{pmatrix}
-1 & -3 \\
2 & 0
\end{pmatrix}, \ ABC = \begin{pmatrix}
-1 & -4 \\
2 & 2
\end{pmatrix}\);
3) \((AB)^3 = \begin{pmatrix}
11 & 15 \\
-10 & 6
\end{pmatrix}, \ C^n = \begin{pmatrix}
1 & n \\
0 & 1
\end{pmatrix}\);
4) \(A^T = \begin{pmatrix}
2 & 0 \\
-1 & 1 \\
3 & 2
\end{pmatrix}, \ A^T C = \begin{pmatrix}
2 & 2 \\
-1 & 0 \\
3 & 5
\end{pmatrix}\).\\
\textbf{Bài số 4.} Cho ma trận \(A = \begin{pmatrix}
1 & -2 & 6 \\
4 & 3 & -8 \\
2 & -2 & 5
\end{pmatrix}\). Tìm ma trận \(X\) sao cho \(3A + 2X = I_3\).

\textit{Đáp số:} \(X = \begin{pmatrix}
-1 & 3 & -9 \\
-6 & -4 & 12 \\
-3 & 3 & -7
\end{pmatrix}\).\\
\textbf{Bài số 5.} Tính các định thức sau:\\

1. \(\begin{vmatrix}
2 & 0 & 1 \\
3 & 2 & -3 \\
-1 & -3 & 5
\end{vmatrix}\)\\
\vspace*{0.5cm}

2. \(\begin{vmatrix}
1 & 0 & 0 \\
3 & 2 & -4 \\
4 & 1 & 3
\end{vmatrix}\)\\
\vspace*{0.5cm}

3. \(\begin{vmatrix}
1 & 2 & 3 & 4 \\
2 & 3 & 4 & 1 \\
3 & 4 & 1 & 2 \\
4 & 1 & 2 & 3
\end{vmatrix}\)\\
\vspace*{0.5cm}

4. \(\begin{vmatrix}
1 & 0 & 2 & a \\
2 & 0 & b & 0 \\
3 & c & 4 & 5 \\
d & 0 & 0 & 0
\end{vmatrix}\)\\
\vspace*{0.5cm}

5. \(\begin{vmatrix}
x & a & b & 0 & c \\
0 & y & 0 & d & 0 \\
0 & 0 & z & 0 & f \\
g & h & k & u & l \\
0 & 0 & 0 & 0 & v
\end{vmatrix}\)
\vspace*{0.5cm}

6. \(\begin{vmatrix}
2 & 1 & 1 & 1 \\
1 & 3 & 1 & 1 \\
1 & 1 & 4 & 1 \\
1 & 1 & 1 & 5 \\
1 & 1 & 1 & 6
\end{vmatrix}\)\\
\vspace*{0.5cm}
\textit{Đáp số:}
1) \(-5\);
2) \(10\);
3) \(160\);
4) \(abcd\);
5) \(xyzuv\);
6) \(394\).\\
\textbf{Bài số 6.} Chứng tỏ rằng các định thức sau bằng không

1. \(\begin{vmatrix}
a + b & c & 1 \\
b + c & a & 1 \\
c + a & b & 1
\end{vmatrix}\)
\vspace*{0.5cm}

2. \(\begin{vmatrix}
x & p & ax + bp \\
y & q & ay + bq \\
z & r & az + br
\end{vmatrix}\)
\vspace*{0.5cm}

3. \(\begin{vmatrix}
    ab & a^2 + b^2 & (a + b)^2 \\
    bc & b^2 + c^2 & (b + c)^2 \\
    ca & c^2 + a^2 & (c + a)^2
    \end{vmatrix}\)
    \vspace*{0.5cm}
    
    4. \(\begin{vmatrix}
    a & b & c & 1 \\
    b & c & a & 1 \\
    c & a & b & 1 \\
    c + b & b + a & a + c & 2
    \end{vmatrix}\)
    \vspace*{0.5cm}
    
    \textit{Hướng dẫn:}
    1) Lấy cột 1 cộng cột 2;
    2) Từ cột 3, ta tách làm hai ma trận có cùng cột 1 và 2;
    3) Lấy cột 2 cộng 2 lần cột 1;
    4) Lấy cột 1 cộng cột 2 và cột 3.\\
\textbf{Bài số 7.} Chứng minh rằng:
    \[
    \begin{vmatrix}
    1 & a & a^2 \\
    1 & b & b^2 \\
    1 & c & c^2
    \end{vmatrix} = (b - a)(c - a)(c - b)
    \]
    
    \textit{Hướng dẫn:} Biến đổi sơ cấp hoặc dùng quy tắc 6 đường chéo.\\
\textbf{Bài số 8.} Tìm \(x\) sao cho:
    \[
    \begin{vmatrix}
    1 & x & x^2 & x^3 \\
    1 & 2 & 4 & 8 \\
    1 & 3 & 9 & 27 \\
    1 & 4 & 16 & 64
    \end{vmatrix} = 0
    \]
    
    \textit{Đáp số:} \(x = 2 \lor x = 3 \lor x = 4\).\\   
\textbf{Bài số 9.} Tính định thức cấp \(n\) sau:
    1. \(\begin{vmatrix}
    1 & 2 & 3 & \cdots & n \\
    -1 & 0 & 0 & \cdots & 0 \\
    -1 & -2 & 0 & \cdots & 0 \\
    \cdots & \cdots & \cdots & \cdots & \cdots \\
    -1 & -2 & -3 & \cdots & 0
    \end{vmatrix}\)
    
    2. \(\begin{vmatrix}
    a & 1 & 1 & \cdots & 1 \\
    1 & a & 1 & \cdots & 1 \\
    1 & 1 & a & \cdots & 1 \\
    \cdots & \cdots & \cdots & \cdots & \cdots \\
    1 & 1 & 1 & \cdots & a
    \end{vmatrix}\)
    \vspace*{0.5cm}

    3. \(\begin{vmatrix}
    1 & 2 & 2 & \cdots & 2 \\
    a_1 + 2b_1 & a_1 + 2b_2 & \cdots & a_1 + 2b_n \\
    a_2 + 2b_1 & a_2 + 2b_2 & \cdots & a_2 + 2b_n \\
    \cdots & \cdots & \cdots & \cdots & \cdots \\
    a_n + 2b_1 & a_n + 2b_2 & \cdots & a_n + 2b_n
    \end{vmatrix}\)
    \vspace*{0.5cm}

    4. \(\begin{vmatrix}
    1 & 0 & 2 & a \\
    2 & 0 & b & 0 \\
    3 & c & 4 & 5 \\
    d & 0 & 0 & 0
    \end{vmatrix}\)
    \vspace*{0.5cm}

    \textit{Đáp số:} 
    1) \(n!\);
    2) \((a + n - 1)(a - 1)^{n-1}\);
    3) ;
    4) \(0\).\\ 
\textbf{Bài số 10.} Cho hai ma trận: \( A = \begin{pmatrix}
    2 & 1 \\
    1 & 2
    \end{pmatrix} \) và \( B = \begin{pmatrix}
    1 & -1 \\
    1 & 1
    \end{pmatrix} \).
    
    Tính \((B^{-1}AB)^n, n \in \mathbb{N}\) rồi suy ra \(A^n\).

    \textit{Đáp số:}
    
    \((B^{-1}AB)^n = \begin{pmatrix}
        3^n & 0 \\
        0 & 1
        \end{pmatrix}; \quad A^n = \frac{1}{2} \begin{pmatrix}
        3^n + 1 & 3^n - 1 \\
        3^n - 1 & 3^n + 1
        \end{pmatrix}\).
        
        
\textbf{Bài số 11.} Cho ma trận \(A = \begin{pmatrix}
        5 & 4 \\
        -4 & -3
        \end{pmatrix} \in M_2 \).
        
        Chứng minh rằng: \(A^2 - 2A + I_2 = 0\). Suy ra \(A^{-1}\).
        
        \textit{Hướng dẫn:} Tính trực tiếp ta có điều phải chứng minh rồi suy ra \(A^{-1}\).\\
\textbf{Bài số 12.} Tìm \(a\) để ma trận sau khả nghịch và tính \(A^{-1}\):
        \[
        A = \begin{pmatrix}
        1 & 1 & 0 \\
        1 & a & 1 \\
        0 & 2 & 1
        \end{pmatrix}
        \]
        
        \textit{Đáp số:} \(a \neq 3; \quad A^{-1} = \frac{1}{a - 3} \begin{pmatrix}
        a - 2 & -1 & 1 \\
        -1 & 1 & -1 \\
        2 & -2 & a - 2
        \end{pmatrix}\).\\     
\textbf{Bài số 13.} Tìm \(m\) sao cho các ma trận sau khả nghịch:
        
        1.
        \[
        \begin{pmatrix}
        1 & 2 & 2 \\
        -2 & m - 2 & m - 5 \\
        m & 1 & m + 1
        \end{pmatrix}
        \]
        
        2.
        \[
        \begin{pmatrix}
        1 & 1 & 1 & m \\
        1 & 1 & 1 & 1 \\
        1 & 1 & 1 & 1 \\
        m & 1 & 1 & 1
        \end{pmatrix}
        \]
        
        \textit{Đáp số:} 1) \(m \neq 1 \land m \neq 3\); 2) \(m \neq 1 \land m \neq -3\).\\
\textbf{Bài số 14.} Tìm \(x\) sao cho:
        \[
        \begin{vmatrix}
        1 & x & x - 1 & x + 2 \\
        0 & x^2 - 1 & 0 & 0 \\
        x & 1 & x & x - 2 \\
        0 & 0 & x^5 + 1 & x^{100}
        \end{vmatrix} = 0
        \]
        
        \textit{Đáp số:} \(x = 0 \lor x = 1 \lor x = -1\).\\ 
\textbf{Bài số 15.} Tìm ma trận nghịch đảo của ma trận sau (nếu có):
        
        1.
        \(
        \begin{pmatrix}
        1 & -1 & 1 \\
        -1 & 2 & 1 \\
        -2 & 3 & 1
        \end{pmatrix}
        \)
        \vspace*{0.5cm}

        2.
        \(
        \begin{pmatrix}
        1 & 2 & -3 \\
        2 & 1 & -2 \\
        2 & -1 & 0
        \end{pmatrix}
        \)
        \vspace*{0.5cm}

        3.
        \(
        \begin{pmatrix}
        0 & 0 & 1 & -1 \\
        0 & 3 & 1 & 4 \\
        -1 & 0 & 0 & 0 \\
        0 & 0 & 1 & 1
        \end{pmatrix}
        \)
        \vspace*{0.5cm}

        4.
        \(
        \begin{pmatrix}
        1 & 1 & 1 & 1 \\
        1 & 1 & -1 & 1 \\
        1 & -1 & 1 & 1 \\
        -1 & 1 & 1 & 1
        \end{pmatrix}
        \)
        \vspace*{0.5cm}

        \textit{Đáp số:}

        1) \(A^{-1} = \begin{pmatrix}
        -1 & 4 & -3 \\
        -1 & 3 & -2 \\
        1 & -1 & 1
        \end{pmatrix}\);\\
        \vspace*{0.5cm}

        2) \(A^{-1} = \begin{pmatrix}
        -2 & 3 & -2 \\
        -2 & 5 & -3 \\
        3 & -2 & 1
        \end{pmatrix}\);\\
        \vspace*{0.5cm}

        3) \(A^{-1} = \begin{pmatrix}
        \frac{1}{2} & \frac{1}{3} & 0 & \frac{1}{5} \\
        1 & \frac{1}{2} & 0 & \frac{1}{6} \\
        1 & \frac{1}{3} & 0 & 0 \\
        1 & \frac{1}{2} & 0 & 0
        \end{pmatrix}\);\\
        \vspace*{0.5cm}

        4) \(A^{-1} = \begin{pmatrix}
        \frac{1}{4} & \frac{1}{4} & \frac{1}{4} & \frac{1}{4} \\
        \frac{1}{4} & \frac{1}{4} & \frac{1}{4} & \frac{1}{4} \\
        \frac{1}{4} & \frac{1}{4} & \frac{1}{4} & \frac{1}{4} \\
        \frac{1}{4} & \frac{1}{4} & \frac{1}{4} & \frac{1}{4}
        \end{pmatrix}\).\\
        \vspace*{0.5cm}\\
\textbf{Bài số 16.} Cho ma trận:
        \[
        A = \begin{pmatrix}
        0.4 & 0.2 & 0.1 \\
        0.1 & 0.3 & 0.4 \\
        0.2 & 0.2 & 0.3
        \end{pmatrix}
        \]
        
        Tìm ma trận \((I - A)^{-1}\).
        
        \textit{Đáp số:}
        \[
        (I - A)^{-1} = \begin{pmatrix}
        2.05 & 0.8 & 0.75 \\
        0.75 & 2 & 1.25 \\
        0.8 & 0.8 & 2
        \end{pmatrix}
        \]\\
\textbf{Bài số 17.} Cho các ma trận:
        \[
        A = \begin{pmatrix}
        -3 & 4 & 6 \\
        0 & 1 & 1 \\
        2 & -3 & -4
        \end{pmatrix}, \quad B = \begin{pmatrix}
        1 & -1 & 2 \\
        0 & 1 & 2
        \end{pmatrix}
        \]
        
        Tìm ma trận \(X\), sao cho \(XA = B\).
        
        \textit{Đáp số:}
        \[
        X = \begin{pmatrix}
        7 & 4 & 11 \\
        2 & 2 & 3
        \end{pmatrix}
        \]\\
\textbf{Bài số 18.} Giải phương trình: \(AX = B\), với:
        \[
A = \begin{pmatrix}
3 & -4 & 5 \\
2 & -3 & 1 \\
3 & -5 & -1
\end{pmatrix}, \quad
B = \begin{pmatrix}
2 & 1 \\
-4 & 3 \\
6 & 5
\end{pmatrix}
\]

\textit{Đáp số:}
\[
X = \begin{pmatrix}
-198 & 24 \\
-124 & 14 \\
20 & -3
\end{pmatrix}
\]\\
\textbf{Bài số 19.} Tìm \(A\) sao cho \(AB = BA\), với
        \[
        A = \begin{pmatrix}
        3 & -4 & 5 \\
        2 & -3 & 1 \\
        3 & -5 & -1
        \end{pmatrix}, \quad B = \begin{pmatrix}
        2 & 1 \\
        -4 & 3 \\
        6 & 5
        \end{pmatrix}
        \]
        
        \textit{Đáp số:}
        \[
        X = \begin{pmatrix}
        -198 & 24 \\
        -124 & 14 \\
        20 & -3
        \end{pmatrix}
        \]\\
\textbf{Bài số 20.} Tính hạng của các ma trận sau:
        
        1.
        \(
        \begin{pmatrix}
        1 & -5 & 4 & 3 & 1 \\
        -2 & -1 & 2 & 1 & 0 \\
        5 & 3 & 1 & 2 & 1 \\
        4 & 9 & 10 & 5 & 2
        \end{pmatrix}
        \)\\
        \vspace*{0.5cm}

        
        2.
        \(
        \begin{pmatrix}
        3 & -1 & -1 & 2 & 1 \\
        1 & -1 & -2 & 4 & 5 \\
        1 & 1 & 3 & -6 & -9 \\
        12 & -2 & 1 & -2 & -10
        \end{pmatrix}
        \)\\
        \vspace*{0.5cm}

        
        3.
        \(
        \begin{pmatrix}
        0 & 1 & -3 & 4 & -5 \\
        1 & 0 & -2 & 3 & -4 \\
        2 & 0 & -3 & 2 & -3 \\
        4 & 0 & -5 & 1 & -5
        \end{pmatrix}
        \)\\
        \vspace*{0.5cm}

        4.
        \(
        \begin{pmatrix}
        1 & -3 & 2 & -1 \\
        -2 & 4 & -3 & 1 \\
        1 & 2 & 1 & 1 \\
        -1 & 2 & -1 & -2
        \end{pmatrix}
        \)\\
        \vspace*{0.5cm}

        
        \textit{Đáp số:}
        1) 3;
        2) 2;
        3) 2;
        4) 4.\\
\textbf{Bài số 21.} Tùy theo \(m\), tìm hạng của các ma trận sau:
        
        1.
        \(
        \begin{pmatrix}
        m & 5m & -m \\
        2m & m & 10m \\
        -m & -2m & -3m
        \end{pmatrix}
        \)\\
        \vspace*{0.5cm}

        
        2.
        \(
        \begin{pmatrix}
        3 & 1 & 1 & 4 \\
        m & 4 & 10 & 1 \\
        1 & 7 & 7 & 3 \\
        2 & 4 & 4 & 3
        \end{pmatrix}
        \)\\
        \vspace*{0.5cm}

        
        3.
        \(
        \begin{pmatrix}
        1 & 2 & 3 & 4 \\
        2 & 3 & 4 & 5 \\
        3 & 4 & 5 & 6 \\
        4 & 5 & 6 & m
        \end{pmatrix}
        \)\\
        \vspace*{0.5cm}

        
        4.
        \(
        \begin{pmatrix}
        -1 & 2 & 1 & -1 \\
        m & -1 & 1 & -1 \\
        1 & m & 0 & 1 \\
        1 & 2 & -1 & 1
        \end{pmatrix}
        \)\\
        \vspace*{0.5cm}

        
        \textit{Đáp số:}
        1) \(m = 0\), rank = 0; \(m \neq 0\), rank = 2;
        2) \(m = 0\), rank = 0; \(m \neq 0\), rank = 3;
        3) \(m = 7\), rank = 2; \(m \neq 7\), rank = 3;
        4) \(m = 1\), rank = 3; \(m \neq 1\), rank = 4.\\
\textbf{Bài số 22*.} Tính \(A^n\), biết rằng:
        
        1.
        \(
        A = \begin{pmatrix}
        \cos x & -\sin x \\
        \sin x & \cos x
        \end{pmatrix}
        \)
        
        2.
        \(
        A = \begin{pmatrix}
        2 & 1 \\
        1 & 2
        \end{pmatrix}
        \)
        
        3.
        \(
        A = \begin{pmatrix}
        4 & 1 \\
        0 & 3
        \end{pmatrix}
        \)
        
        4.
        \(
        A = \begin{pmatrix}
        \frac{\sqrt{3} + 1}{2} & -1 \\
        \frac{1}{2} & \frac{\sqrt{3} - 1}{2}
        \end{pmatrix}
        \)
        
        \textit{Đáp số:}

        1) \(A^n = \begin{pmatrix}
        \cos nx & -\sin nx \\
        \sin nx & \cos nx
        \end{pmatrix}\);
        
        2) \(A^n = \frac{1}{2} \begin{pmatrix}
        3^n + 1 & 3^n - 1 \\
        3^n - 1 & 3^n + 1
        \end{pmatrix}\);
        
        3) \(A^n = \begin{pmatrix}
        4^n & 4^n - 3^n \\
        0 & 3^n
        \end{pmatrix}\);
        
        4) \(A^n = \begin{pmatrix}
        \cos \frac{n\pi}{6} + \sin \frac{n\pi}{6} & -2\sin \frac{n\pi}{6} \\
        \sin \frac{n\pi}{6} & \cos \frac{n\pi}{6} - \sin \frac{n\pi}{6}
        \end{pmatrix}\).\\
\textbf{Bài số 23*.} Tìm \(a, b\) sao cho:
        \[
        \begin{pmatrix}
        a & -b \\
        b & a
        \end{pmatrix}^4 = \begin{pmatrix}
        \sqrt{3} & -1 \\
        1 & \sqrt{3}
        \end{pmatrix}
        \]
        
        \textit{Đáp số:}
        \[
        a = \frac{\sqrt{2}}{4} \cos \left( \frac{\pi}{24} + k \frac{\pi}{2} \right), \quad b = \frac{\sqrt{2}}{4} \sin \left( \frac{\pi}{24} + k \frac{\pi}{2} \right).
        \]\\
\textbf{Bài số 24*.} Cho hai ma trận:
        \[
        A = \begin{pmatrix}
        2 & 0 & 0 \\
        1 & 1 & 0 \\
        0 & 0 & 2
        \end{pmatrix}, \quad B = \begin{pmatrix}
        2 & 1 & 0 \\
        0 & 1 & 0 \\
        0 & 0 & 2
        \end{pmatrix}
        \]
        
        Chứng minh rằng \(\det(A^n + B^n)\) chia hết cho \(2^{n+1}\).
        
        \textit{Hướng dẫn:}
        \[
        A = \begin{pmatrix}
        1 & 0 & 0 \\
        0 & 1 & 0 \\
        0 & 0 & 1
        \end{pmatrix} + \begin{pmatrix}
        1 & 0 & 0 \\
        1 & 0 & 0 \\
        0 & 0 & 1
        \end{pmatrix}, \quad B = \begin{pmatrix}
        1 & 0 & 0 \\
        0 & 1 & 0 \\
        0 & 0 & 1
        \end{pmatrix} + \begin{pmatrix}
        1 & 1 & 0 \\
        0 & 0 & 0 \\
        0 & 0 & 1
        \end{pmatrix}
        \]\\
\textbf{Bài số 25*.} Cho \(A, B, C\) là ba ma trận vuông cấp 2 với các phần tử của ma trận là số thực. Chứng minh rằng:
        \[
        (AB - BA)^2C - C(AB - BA)^2 = 0.
        \]
        
        \textit{Hướng dẫn:} Đặt \(A, B, C\) rồi đi tính trực tiếp ta có điều phải chứng minh.
\newpage
\begin{thebibliography}{9}
    \bibitem{toancaocap}
    Nguyễn Huy Hoàng, Nguyễn Trung Đông,
    {\em Giáo Trình Toán Cao Cấp}
    % Bộ Giáo Dục và Đào Tạo, 
    % Tái bản lần thứ mười một, 2024.
    % \bibitem{ttdeo}
    % Trịnh Thanh Đèo,
    % {\em Soạn thảo và chế bản tài liệu toán học với \LaTeX2$\varepsilon$},
    % NXB Đại~học Quốc gia TP.HCM, 2006.
    % \bibitem{johson}
    % Johnson Richard, Wichern Dean,
    % {\em Applied Multivariate Statistical Analysis}, tái bản lần thứ tư, Springer-Verlag, 2013.
\end{thebibliography}