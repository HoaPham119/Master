\documentclass[a4paper]{book}
\usepackage[vietnamese]{babel}
\usepackage[fontsize=13pt]{fontsize}
\usepackage[T5]{fontenc}
\usepackage[utf8]{inputenc}
\usepackage{amsmath}
\usepackage{graphicx}
\usepackage{geometry}
\usepackage{fancyhdr}
\usepackage{titlesec}
\usepackage{mdframed}
\usepackage{setspace}
\usepackage{times} % Sử dụng font Times New Roman
\usepackage{tikz}
\usetikzlibrary{calc}

% Định dạng trang
\geometry{
  a4paper,
  left=3.5cm,
  right=2cm,
  top=3.5cm,
  bottom=3cm
}

% Cỡ chữ và giãn dòng
\setstretch{1.15} % Tương đương với giãn dòng 1.5 trong Winword

% % Định dạng tiêu đề chương
% \titleformat{\chapter}[display]
%   {\normalfont\huge\bfseries}{\chaptertitlename\ \thechapter}{20pt}{\Huge}

% Định dạng đầu trang và cuối trang
\fancypagestyle{plain}{
  \fancyhf{}
  \fancyfoot[C]{\thepage}
  \renewcommand{\headrulewidth}{0pt}
  \renewcommand{\footrulewidth}{0pt}
}
\pagestyle{plain}

\begin{document}

% Trang bìa
\begin{titlepage}
    % \begin{mdframed}[linewidth=1.5pt, linecolor=black, outermargin=3cm, innerrightmargin=2cm, innertopmargin=3.5cm, innerbottommargin=3pt, ]
    % \begin{mdframed}[linewidth=1pt]
    \tikz[overlay, remember picture]
    \draw ($(current page.north west) + (3cm,-2.5cm)$)
            rectangle ($(current page.south east) + (-2.5cm,2cm)$);
    \begin{center}
        % \centerline{Phạm Thị Hoà}
        \fontsize{14}{16}\selectfont
        ĐẠI HỌC QUỐC GIA TP. HCM\\
        \textbf{TRƯỜNG ĐẠI HỌC KHOA HỌC TỰ NHIÊN}
       
        \vspace{3cm}
        
        \fontsize{14}{16}\selectfont
        \textbf{PHẠM THỊ HOÀ}
        
        \vspace{3cm}
        
        \fontsize{16}{18}\selectfont
        \textbf{THỰC HÀNH THIẾT KẾ LUẬN VĂN BẰNG LATEX}
        
        \vspace{5cm}
        
        \fontsize{14}{16}\selectfont
        \textbf{LUẬN VĂN THẠC SĨ}
        
        \vspace{7cm}
        
        \fontsize{12}{14}\selectfont
        TP. Hồ Chí Minh - Năm 2024
    \end{center}
\end{titlepage}

% trang phụ bìa luận văn
\begin{titlepage}
    \tikz[overlay, remember picture]
    \draw ($(current page.north west) + (3cm,-2.5cm)$)
            rectangle ($(current page.south east) + (-2.5cm,2cm)$);
    % \begin{mdframed}[linewidth=1.5pt, linecolor=black, outermargin=3cm, innerrightmargin=2cm, innertopmargin=3.5cm, innerbottommargin=3pt, ]
    %\begin{mdframed}[linewidth=1pt]
        \tikz[overlay, remember picture]
        \draw ($(current page.north west) + (3cm,-2.5cm)$)
              rectangle ($(current page.south east) + (-2.5cm,2cm)$);
        \begin{center}
        \fontsize{13}{16}\selectfont
        ĐẠI HỌC QUỐC GIA TP. HCM\\
        \textbf{TRƯỜNG ĐẠI HỌC KHOA HỌC TỰ NHIÊN}
        
        \vspace{2cm}
        
        \fontsize{14}{16}\selectfont
        \textbf{PHẠM THỊ HOÀ}
        
        \vspace{2cm}
        
        \fontsize{16}{18}\selectfont
        \textbf{THỰC HÀNH THIẾT KẾ LUẬN VĂN BẰNG LATEX}
        
        \vspace{5cm}
        
        \fontsize{13}{16}\selectfont
        \raggedright{Ngành: Lý thuyết xác suất và thống kê toán học}\\
        Mã số Ngành: 8460106
        
        \vspace{3cm}
        
        \fontsize{12}{14}\selectfont
        NGƯỜI HƯỚNG DẪN KHOA HỌC\\
        \raggedright{PGS.TS: Trịnh Thanh Đèo}
    \end{center}
    % \end{mdframed}
\end{titlepage}

% Phần nội dung chính
\tableofcontents
\chapter*{Tiêu đề}
\addcontentsline{toc}{chapter}{Tiêu đề}
\begin{center}
    \Large
    \textbf{THỰC HÀNH THIẾT KẾ LUẬN VĂN BẰNG LATEX}
\end{center}

\vspace{1cm}

\begin{flushleft}
    \large
    \textbf{Tên tác giả:} \\
    \textbf{Địa chỉ:} (Gồm mã số học viên, lớp, khóa)
\end{flushleft}

\vspace{1cm}

\chapter{Chương 1: Đại số, Giải tích, Xác suất, Thống kê}
\section{Mục 1}
Nội dung mục 1 thuộc chương 1.

\section{Mục 2}
Nội dung mục 2 thuộc chương 1.

\chapter{Chương 2: Hình học}
\section{Mục 1}
Nội dung mục 1 thuộc chương 2. Bao gồm hình vẽ minh họa.

\begin{figure}[h]
    \centering
    \includegraphics[width=0.5\textwidth]{example-image} % Thay example-image bằng đường dẫn đến hình ảnh của bạn
    \caption{Hình minh họa cho mục 1 chương 2}
    \label{fig:example}
\end{figure}

\section{Mục 2}
Nội dung mục 2 thuộc chương 2.

\chapter*{Tài liệu tham khảo}
\addcontentsline{toc}{chapter}{Tài liệu tham khảo}
\begin{itemize}
    \item [1.] Phạm Thị Thu Hà (2014). Giáo trình Quản lý dự án, NXB Bách khoa Hà Nội.
    % \item [2.] Tran Thi Bich Ngoc et.al. (2016). The Care of Elderly People in Vietnam, European Proceedings of Social and Behavioural Sciences, 7, 485-501.
    \item [3.] Nguyễn Danh Nguyên, Phạm Thị Thanh Hồng (2016). Mô hình sản xuất hiệu suất cao: Đặc điểm và vai trò đối với sự phát triển của nền kinh tế nói chung, Tạp chí Quản lý kinh tế, 75, 73-79.
    \item [4.] Trinh Thu Thuy, Pham Thi Thanh Hong, and Mai Fujita (2016). Supporting Industries in Vietnam: Situation and Determinants, Proceedings of International Conference on Emerging Challenges: Partnership Enhancement, 1, 3-16.
    % \item [5.] World Bank (2016) World Development Indicators Online, http://publications.worldbank/WDI/ truy cập ngày 17/7/2016.
\end{itemize}

\end{document}
