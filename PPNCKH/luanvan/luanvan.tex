\documentclass[a4paper]{book}
\usepackage[vietnamese]{babel}
% \usepackage[fontsize=13pt]{fontsize}
\changefontsize{13pt}
\usepackage[T5]{fontenc}
\usepackage[utf8]{inputenc}
\usepackage{amsmath}
\usepackage{graphicx}
\usepackage{geometry}
\usepackage{fancyhdr}
\usepackage{titlesec}
\usepackage{mdframed}
\usepackage{setspace}
\usepackage{pgfplots}
\usepackage{mathrsfs}
\usepackage{amsfonts}
\usepackage{amssymb}
\usepackage{enumitem}
\usepackage{tcolorbox}
\usepackage{times} % Sử dụng font Times New Roman
\usepackage{tikz}
\usetikzlibrary{calc}

\usepackage{mathrsfs}
\usetikzlibrary{arrows}
\pagestyle{empty}
% Định dạng trang
\geometry{
  a4paper,
  left=3.5cm,
  right=2cm,
  top=3.5cm,
  bottom=3cm
}

% Cỡ chữ và giãn dòng
\setstretch{1.15} % Tương đương với giãn dòng 1.5 trong Winword

% % Định dạng tiêu đề chương
% \titleformat{\chapter}[display]
%   {\normalfont\huge\bfseries}{\chaptertitlename\ \thechapter}{20pt}{\Huge}

% Định dạng đầu trang và cuối trang
\fancypagestyle{plain}{
  \fancyhf{}
  \fancyfoot[C]{\thepage}
  \renewcommand{\headrulewidth}{0pt}
  \renewcommand{\footrulewidth}{0pt}
}
\pagestyle{plain}

\begin{document}

% Trang bìa
\newgeometry{top=2.5cm, bottom=2cm, left=3cm, right=2.5cm}
\begin{titlepage}
    % \begin{mdframed}[linewidth=1.5pt, linecolor=black, outermargin=3cm, innerrightmargin=2cm, innertopmargin=3.5cm, innerbottommargin=3pt, ]
    % \begin{mdframed}[linewidth=1pt]
    \tikz[overlay, remember picture]
    \draw ($(current page.north west) + (3cm,-2.5cm)$)
            rectangle ($(current page.south east) + (-2.5cm,2cm)$);
    \begin{center}
        % \centerline{Phạm Thị Hoà}
        \fontsize{14}{16}\selectfont
        ĐẠI HỌC QUỐC GIA TP. HCM\\
        \textbf{TRƯỜNG ĐẠI HỌC KHOA HỌC TỰ NHIÊN}
       
        \vspace{3cm}
        
        \fontsize{14}{16}\selectfont
        \textbf{PHẠM THỊ HOÀ}
        
        \vspace{3cm}
        
        \fontsize{16}{18}\selectfont
        \textbf{THỰC HÀNH THIẾT KẾ LUẬN VĂN BẰNG LATEX}
        
        \vspace{5cm}
        
        \fontsize{14}{16}\selectfont
        \textbf{LUẬN VĂN THẠC SĨ}
        
        \vspace{9.5cm}
        
        \fontsize{12}{14}\selectfont
        TP. Hồ Chí Minh - Năm 2024
    \end{center}
\end{titlepage}

% trang phụ bìa luận văn
\newgeometry{top=2cm, bottom=2cm, left=3.5cm, right=2.5cm}
\begin{titlepage}
    \tikz[overlay, remember picture]
    \draw ($(current page.north west) + (3.5cm,-2.5cm)$)
            rectangle ($(current page.south east) + (-2.5cm,2cm)$);
        \begin{center}
        \fontsize{13}{16}\selectfont
        ĐẠI HỌC QUỐC GIA TP. HCM\\
        \textbf{TRƯỜNG ĐẠI HỌC KHOA HỌC TỰ NHIÊN}
        
        \vspace{2cm}
        
        \fontsize{14}{16}\selectfont
        \text{PHẠM THỊ HOÀ}
        
        \vspace{2cm}
        
        \fontsize{16}{18}\selectfont
        \textbf{THỰC HÀNH THIẾT KẾ LUẬN VĂN BẰNG LATEX}
        
        \vspace{5.5cm}
        
    \end{center}

        \fontsize{13}{16}\selectfont
        
        \hspace{1cm}Ngành: Lý thuyết xác suất và thống kê toán học
        \par
        \hspace{1cm}Mã số Ngành: 8460106\\
        
        \vspace{3cm}
        
        \fontsize{12}{14}\selectfont
        \hspace{1cm}NGƯỜI HƯỚNG DẪN KHOA HỌC
        \par
        \hspace{2cm}TS: Trịnh Thanh Đèo

        \vspace{5cm}
        
        \begin{center}
        \fontsize{12}{14}\selectfont
        TP. Hồ Chí Minh - Năm 2024
        \end{center}
    % \end{mdframed}
\end{titlepage}

% quay trở về căn lề chính
\restoregeometry

% Phần nội dung chính
\tableofcontents
\chapter{MÔ HÌNH NHÂN TỐ TRỰC GIAO}

Vector ngẫu nhiên \(X\) được quan sát với các giá trị:
\begin{itemize}
\item Số thành phần: \(p\) ,

\item Trung bình: \(\mu\),

\item Ma trận hiệp phương sai: \(\Sigma\).
\end{itemize}
Mô hình nhân tố giả định rằng: \(X\) phụ thuộc tuyến tính vào một số lượng ít các nhân tố không quan sát được:
\begin{itemize}
\item \(F_1, F_2, \ldots, F_m\), được gọi là nhân tố chung (common factors).

\item \(p\) sai số  \(\boldsymbol{\epsilon}_1, \boldsymbol{\epsilon}_2, \ldots, \boldsymbol{\epsilon}_p\).
\end{itemize}
Mô hình phân tích nhân tố được mô tả:
\begin{eqnarray}
X_1 - \mu_1 &= \ell_{11}F_1 + \ell_{12}F_2 + \ldots + \ell_{1m}F_m + \boldsymbol{\epsilon}_1\\
X_2 - \mu_2 &= \ell_{21}F_1 + \ell_{22}F_2 + \ldots + \ell_{2m}F_m + \boldsymbol{\epsilon}_2
\end{eqnarray}
\[\vdots\]
\begin{eqnarray}
X_p - \mu_p &= \ell_{p1}F_1 + \ell_{p2}F_2 + \ldots + \ell_{pm}F_m + \boldsymbol{\epsilon}_p
\end{eqnarray}
hoặc, dưới dạng ký hiệu ma trận:
\begin{eqnarray}
\mathbf{X} - \boldsymbol{\mu} = \mathbf{L}\mathbf{F} + \boldsymbol{\varepsilon}
\end{eqnarray}

Trong đó: \(\ell_{ij}\) được gọi là hệ số tải (loading) của biến thứ \(i\) trên nhân tố thứ \(j\), do đó ma trận \(L\) là ma trận của các hệ số tải nhân tố (matrix of factor loadings).

Lưu ý: 
\begin{itemize}
    \item Sai số thứ \(i\) \(\boldsymbol{\epsilon}_i\) chỉ liên quan đến phản hồi \(X_i\).
    \item \(p\) độ lệch : \(X_1 - \mu_1, X_2 - \mu_2, \ldots, X_p - \mu_p\) được biểu diễn bằng \(p + m\) theo các biến ngẫu nhiên  \(F_1, F_2, \ldots, F_m, \boldsymbol{\epsilon}_1, \boldsymbol{\epsilon}_2, \ldots, \boldsymbol{\epsilon}_p\), với \(F_i\) và \(\boldsymbol{\epsilon}_i\) không thể quan sát được.
\end{itemize}
%Điều này phân biệt mô hình nhân tố trong (9-2) với mô hình hồi quy đa biến trong (7-23), trong đó biến độc lập [vị trí của nó được chiếm bởi \(F\) trong (9-2)] có thể được quan sát.

Với rất nhiều đại lượng không quan sát được, việc xác minh trực tiếp mô hình nhân tố từ các quan sát trên \(X_1, X_2, \ldots, X_p\) là không thể. Tuy nhiên, với một số giả định bổ sung về các vector ngẫu nhiên \(F\) và \(\varepsilon\), mô hình nhân tố trực giao đưa ra một số mối quan hệ hiệp phương sai, mà có thể được kiểm tra.

Chúng ta giả định rằng:
\begin{eqnarray}
E(\mathbf{F}) = \mathbf{0}_{m \times 1}, \quad \text{Cov}(\mathbf{F}) = E[\mathbf{FF}'] = \mathbf{I}_{m \times m},
\end{eqnarray}
\begin{eqnarray}
E(\boldsymbol{\varepsilon}) = \mathbf{0}_{p \times 1}, \quad \text{Cov}(\boldsymbol{\varepsilon}) = E[\boldsymbol{\varepsilon} \boldsymbol{\varepsilon}'] = \boldsymbol{\Psi} =
\begin{bmatrix}
\psi_1 & 0 & \cdots & 0 \\
0 & \psi_2 & \cdots & 0 \\
\vdots & \vdots & \ddots & \vdots \\
0 & 0 & \cdots & \psi_p
\end{bmatrix}_{p \times p},
\end{eqnarray}
(công thức 9-3)

Trong đó \(\mathbf{0}_{m \times 1}\) và \(\mathbf{0}_{p \times 1}\) là các vector không, \(\mathbf{I}_{m \times m}\) là ma trận đơn vị kích thước \(m \times m\), và \(\boldsymbol{\Psi}\) là ma trận đường chéo với các phần tử \(\psi_1, \psi_2, \ldots, \psi_p\) trên đường chéo chính.

Do \(F\) và \(\boldsymbol{\epsilon}\) là độc lập, vì vậy:

\begin{eqnarray}
\text{Cov}(\boldsymbol{\epsilon}, \mathbf{F}) = E(\boldsymbol{\epsilon}\mathbf{F}') = \mathbf{0}_{p \times m},
\end{eqnarray}

Những giả định này và mối quan hệ trong (công thức 9-2) tạo thành mô hình nhân tố trực giao (orthogonal factor model).



\chapter{HÌNH HỌC}
\section{Thể tích khối lăng trụ}
Nếu ta xem khối hộp chữ nhật $ABCD.A'B'C'D'$ như là khối lăng trụ có đáy là hình chữ nhật $A'B'C'D'$ và đường cao $AA'$ thì thể tích của nó bằng diện tích đáy nhân với chiều cao. Ta có thể chứng minh được rằng điều đó cũng đúng đối với một khối lăng trụ bất kì.

Thể tích khối lăng trụ có diện tích đáy $B$ và chiều cao $h$ là:
\begin{eqnarray}
    V = Bh
\end{eqnarray}
\definecolor{aqaqaq}{rgb}{0.6274509803921569,0.6274509803921569,0.6274509803921569}
\begin{tikzpicture}[line cap=round,line join=round,>=triangle 45,x=0.3cm,y=0.3cm]
% \clip(-34.46178,-21.823355714077497) rectangle (23.554536323184955,26.839702526695376);
% \fill[line width=2.pt,color=aqaqaq,fill=aqaqaq,fill opacity=0.5] (-22.,-6.) -- (-26.,-8.) -- (-14.,-8.) -- (-10.,-6.) -- cycle;
% \fill[line width=2.pt,color=aqaqaq,fill=aqaqaq,fill opacity=0.75] (8.,-2.) -- (2.,-6.) -- (4.,-8.) -- (10.,-6.) -- (12.,-4.) -- cycle;
\draw [line width=1.2pt] (-22.,4.)-- (-26.,2.);
\draw [line width=1.2pt] (-26.,2.)-- (-14.,2.);
\draw [line width=1.2pt] (-14.,2.)-- (-10.,4.);
\draw [line width=1.2pt] (-10.,4.)-- (-22.,4.);
\draw [line width=1.2pt,dash pattern=on 5pt off 5pt] (-22.,-6.)-- (-26.,-8.);
\draw [line width=1.2pt] (-26.,-8.)-- (-14.,-8.);
\draw [line width=1.2pt] (-14.,-8.)-- (-10.,-6.);
\draw [line width=1.2pt,dash pattern=on 5pt off 5pt] (-10.,-6.)-- (-22.,-6.);
\draw [line width=1.2pt,dash pattern=on 5pt off 5pt] (-22.,4.)-- (-22.,-6.);
\draw [line width=1.2pt] (-26.,2.)-- (-26.,-8.);
\draw [line width=1.2pt] (-14.,2.)-- (-14.,-8.);
\draw [line width=1.2pt] (-10.,4.)-- (-10.,-6.);
\draw [line width=2.pt,color=aqaqaq] (-22.,-6.)-- (-26.,-8.);
\draw [line width=2.pt,color=aqaqaq] (-26.,-8.)-- (-14.,-8.);
\draw [line width=2.pt,color=aqaqaq] (-14.,-8.)-- (-10.,-6.);
\draw [line width=2.pt,color=aqaqaq] (-10.,-6.)-- (-22.,-6.);
\draw [line width=1.2pt] (4.,4.)-- (6.,2.);
\draw [line width=1.2pt] (6.,2.)-- (12.,4.);
\draw [line width=1.2pt] (12.,4.)-- (14.,6.);
\draw [line width=1.2pt] (14.,6.)-- (10.,8.);
\draw [line width=1.2pt] (10.,8.)-- (4.,4.);
\draw [line width=1.2pt] (4.,4.)-- (2.,-6.);
\draw [line width=1.2pt] (2.,-6.)-- (4.,-8.);
\draw [line width=1.2pt] (4.,-8.)-- (6.,2.);
\draw [line width=1.2pt] (12.,4.)-- (10.,-6.);
\draw [line width=1.2pt] (10.,-6.)-- (4.,-8.);
\draw [line width=1.2pt] (14.,6.)-- (12.,-4.);
\draw [line width=1.2pt] (12.,-4.)-- (10.,-6.);
\draw [line width=1.2pt,dash pattern=on 5pt off 5pt] (12.,-4.)-- (8.,-2.);
\draw [line width=1.2pt,dash pattern=on 5pt off 5pt] (8.,-2.)-- (2.,-6.);
\draw [line width=1.2pt,dash pattern=on 5pt off 5pt] (10.,8.)-- (8.,-2.);
\draw [line width=1.2pt,dash pattern=on 5pt off 5pt] (6.,2.)-- (6.,-6.);
\draw [line width=2.pt,color=aqaqaq] (8.,-2.)-- (2.,-6.);
\draw [line width=2.pt,color=aqaqaq] (2.,-6.)-- (4.,-8.);
\draw [line width=2.pt,color=aqaqaq] (4.,-8.)-- (10.,-6.);
\draw [line width=2.pt,color=aqaqaq] (10.,-6.)-- (12.,-4.);
\draw [line width=2.pt,color=aqaqaq] (12.,-4.)-- (8.,-2.);
\begin{scriptsize}
\draw [fill=black] (-22.,4.) circle (0.5pt);
\draw[color=black] (-21.511125397981257,4.699319173870623) node {$A$};
\draw [fill=black] (-26.,2.) circle (0.5pt);
\draw[color=black] (-25.631788225896308,3.5219871196583754) node {$B$};
\draw [fill=black] (-14.,2.) circle (0.5pt);
\draw[color=black] (-14.381724632223468,3.325765110623001) node {$C$};
\draw [fill=black] (-10.,4.) circle (0.5pt);
\draw[color=black] (-9.541580993085153,4.699319173870623) node {$D$};
\draw [fill=black] (-22.,-6.) circle (0.5pt);
\draw[color=black] (-20.98786662618252,-4.7847445961724775) node {$A'$};
\draw [fill=black] (-26.,-8.) circle (0.5pt);
\draw[color=black] (-26.87452780891831,-7.662667395357969) node {$B'$};
\draw [fill=black] (-14.,-8.) circle (0.5pt);
\draw[color=black] (-12.22328219855368,-7.989704077083594) node {$C'$};
\draw [fill=black] (-10.,-6.) circle (0.5pt);
\draw[color=black] (-9.083729567761258,-4.653929923482228) node {$D'$};
\draw [fill=black] (4.,4.) circle (0.5pt);
\draw[color=black] (3.670702994832952,5.418799873666996) node {$E$};
\draw [fill=black] (6.,2.) circle (0.5pt);
\draw[color=black] (6.41781154677632,3.1949504379327514) node {$F$};
\draw [fill=black] (12.,4.) circle (0.5pt);
\draw[color=black] (10.865511107065583,5.0917631919413715) node {$G$};
\draw [fill=black] (14.,6.) circle (0.5pt);
\draw[color=black] (14.462915163181899,6.661539264224367) node {$H$};
\draw [fill=black] (10.,8.) circle (0.5pt);
\draw[color=black] (7.595143783323477,9.604869399754985) node {$I$};
\draw [fill=black] (2.,-6.) circle (0.5pt);
\draw[color=black] (1.6430752541128464,-4.457707914446853) node {$E'$};
\draw [fill=black] (4.,-8.) circle (0.5pt);
\draw[color=black] (4.782627884905268,-9.036221458605592) node {$F'$};
\draw [fill=black] (10.,-6.) circle (0.5pt);
\draw[color=black] (10.538474374691372,-6.681557350181097) node {$G'$};
\draw [fill=black] (12.,-4.) circle (0.5pt);
\draw[color=black] (12.631509461886319,-3.345783196579731) node {$H'$};
\draw [fill=black] (8.,-2.) circle (0.5pt);
\draw[color=black] (9.164920098719689,-1.252748433535737) node {$I'$};
\draw [fill=black] (6.,-6.) circle (0.5pt);
\draw[color=black] (7.006477665049899,-5.177188614243226) node {$X$};
\draw[color=black] (7.006477665049899,-1.7760071242967355) node {$h$};
\end{scriptsize}
\end{tikzpicture}

\section{Thể tích khối chóp}
Đối với khối chóp, người ta chứng minh được thể tích của khối chóp có diện tích đáy \( B \) và chiều cao \( h \) là:
\begin{eqnarray}
    V = \frac{1}{3} Bh
\end{eqnarray}
Ta cũng gọi thể tích các khối đa diện, khối lăng trụ, khối chóp lần lượt là thể tích các hình đa diện, hình lăng trụ, hình chóp xác định chúng.

\section*{Ví dụ}
Cho hình lăng trụ tam giác $ABC.A'B'C'$. Gọi $E$ và $F$ lần lượt là trung điểm của các cạnh $AA'$ và $BB'$. Đường thẳng $CE$ cắt đường thẳng $C'A'$ tại $E'$. Đường thẳng $CF$ cắt đường thẳng $C'B'$ tại $F'$. Gọi $V$ là thể tích khối lăng trụ $ABC.A'B'C'$.

\begin{enumerate}[label=(\alph*)]
    \item Tính thể tích khối chóp $C.ABFE$ theo $V$.
    \item Gọi khối đa diện $(H)$ là phần còn lại của khối lăng trụ $ABC.A'B'C'$ sau khi cắt bỏ đi khối chóp $C.ABFE$. Tính tỉ số thể tích của $(H)$ và của khối chóp $C.C'E'F'$.
\end{enumerate}

\section*{Giải}
\begin{tikzpicture}[line cap=round,line join=round,>=triangle 45,x=0.012cm,y=0.012cm]
    % \clip(-688.9650857050052,-703.5746060309538) rectangle (312.48744227387573,97.82139895237928);
    \draw [line width=1.2pt] (-200.,-50.)-- (-200.,-250.);
    \draw [line width=1.2pt] (100.,-50.)-- (-200.,-50.);
    \draw [line width=1.2pt] (-100.,-100.)-- (-200.,-50.);
    \draw [line width=1.2pt] (100.,-50.)-- (-100.,-100.);
    \draw [line width=1.2pt] (-100.,-100.)-- (-100.,-300.);
    \draw [line width=1.2pt] (-100.,-500.)-- (-100.,-300.);
    \draw [line width=1.2pt] (100.,-450.)-- (-100.,-500.);
    \draw [line width=1.2pt,dash pattern=on 3pt off 3pt] (-200.,-450.)-- (-100.,-500.);
    \draw [line width=1.2pt] (100.,-450.)-- (100.,-50.);
    \draw [line width=1.2pt,dash pattern=on 3pt off 3pt] (-498.59181819578856,-450.44996720831983)-- (100.,-450.);
    \draw [line width=1.2pt,dash pattern=on 3pt off 3pt] (-200.,-250.)-- (-200.,-450.);
    \draw [line width=1.2pt] (-200.,-250.)-- (-100.,-300.);
    \draw [line width=1.2pt] (100.,-50.)-- (-100.,-300.);
    \draw [line width=1.2pt] (-300.,-550.)-- (-498.59181819578856,-450.44996720831983);
    \draw [line width=1.2pt] (-300.,-550.)-- (-100.,-300.);
    \draw [line width=1.2pt] (-300.,-550.)-- (-100.,-500.);
    \draw [line width=1.2pt,dash pattern=on 3pt off 3pt] (100.,-50.)-- (-200.,-250.);
    \draw [line width=1.2pt] (-200.,-250.)-- (-498.59181819578856,-450.44996720831983);
    \begin{scriptsize}
    \draw [fill=black] (-200.,-50.) circle (0.5pt);
    \draw[color=black] (-224.50637821947285,-45.49394500449414) node {$A$};
    \draw [fill=black] (-100.,-100.) circle (0.5pt);
    \draw[color=black] (-100.49473339965814,-75.91189555860605) node {$B$};
    \draw [fill=black] (100.,-50.) circle (0.5pt);
    \draw[color=black] (115.94068444624486,-50.173629705126736) node {$C$};
    \draw [fill=black] (-200.,-250.) circle (0.5pt);
    \draw[color=black] (-223.33645704192742,-236.19109655527265) node {$E$};
    \draw [fill=black] (-200.,-450.) circle (0.5pt);
    \draw[color=black] (-224.50637821947285,-430.3980116315256) node {$A'$};
    \draw [fill=black] (-100.,-500.) circle (0.5pt);
    \draw[color=black] (-86.45567926911309,-512.2924938925961) node {$B'$};
    \draw [fill=black] (100.,-450.) circle (0.5pt);
    \draw[color=black] (117.11060562379028,-439.7573810327908) node {$C'$};
    \draw [fill=black] (-100.,-300.) circle (0.5pt);
    \draw[color=black] (-114.53378753020321,-272.4586529851753) node {$F$};
    \draw [fill=black] (-498.59181819578856,-450.44996720831983) circle (0.5pt);
    \draw[color=black] (-504.1175396528286,-437.4175386824745) node {$E'$};
    \draw [fill=black] (-300.,-550.) circle (0.5pt);
    \draw[color=black] (-281.8325159191985,-561.4291832492385) node {$F'$};
    \end{scriptsize}
    \end{tikzpicture}
\begin{enumerate}[label=(\alph*)]
    \item Hình chóp $C.A'B'C'$ và hình lăng trụ $ABC.A'B'C'$ có đáy và đường cao bằng nhau nên $V_{C.A'B'C'} = \frac{1}{3} V$. Từ đó suy ra $V_{C.ABB'A'} = V - \frac{1}{3}V = \frac{2}{3} V$.
    
    Do $EF$ là đường trung bình của hình bình hành $ABB'A'$, nên diện tích $ABFE$ bằng nửa diện tích $ABB'A'$. Do đó $V_{C.ABFE} = \frac{1}{2} V_{C.ABB'A'} = \frac{1}{3} V$.
    
    \item Áp dụng câu a, ta có $V_{(H)} = V_{ABC.A'B'C'} - V_{C.ABFE} = V - \frac{1}{3} V = \frac{2}{3} V$.
    
    Vì $EA'$ song song và bằng $\frac{1}{2} CC'$ nên theo định lý Ta-lét, $A'$ là trung điểm của $E'C'$. Tương tự, $B'$ là trung điểm của $F'C'$. Do đó, diện tích tam giác $C'E'F'$ gấp bốn lần diện tích tam giác $A'B'C'$. Từ đó suy ra $V_{C.E'F'C'} = 4 V_{C.A'B'C'} = \frac{4}{3} V$.
    
    Do đó $\frac{V_{(H)}}{V_{C.E'F'C'}} = \frac{1}{2}$.
\end{enumerate}

\begin{thebibliography}{9}
    \bibitem{hinhhoc12}
Trần Văn Hạo, Nguyễn Mộng Hy, Khu Quốc Anh, Trần Đức Huyến,
\textit{Hình Học 12},
Bộ Giáo Dục và Đào Tạo, 
Tái bản lần thứ mười một, 2024.
\end{thebibliography}


\end{document}
