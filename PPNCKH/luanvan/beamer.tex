\documentclass{beamer}
\usepackage[vietnamese]{babel}
\usepackage[utf8]{inputenc}
\usepackage{amsmath}
\usepackage{graphicx}

\usetheme{Madrid}

\title[Thực hành thiết kế luận văn bằng \LaTeX]{Thực hành thiết kế luận văn bằng \LaTeX}
\author[Phạm Thị Hoà]{Phạm Thị Hoà}
\institute[ĐHQG-HCM]{Đại học Quốc gia TP. HCM}
\date[2024]{TP. Hồ Chí Minh - Năm 2024}

\begin{document}

\begin{frame}
  \titlepage
\end{frame}

\begin{frame}{Mô hình nhân tố trực giao}
Vector ngẫu nhiên \(X\) được quan sát với các giá trị:
\begin{itemize}
\item Số thành phần: \(p\) ,
\item Trung bình: \(\mu\),
\item Ma trận hiệp phương sai: \(\Sigma\).
\end{itemize}
Mô hình nhân tố giả định rằng: \(X\) phụ thuộc tuyến tính vào một số lượng ít các nhân tố không quan sát được:
\begin{itemize}
\item \(F_1, F_2, \ldots, F_m\), được gọi là nhân tố chung (common factors).
\item \(p\) sai số  \(\boldsymbol{\epsilon}_1, \boldsymbol{\epsilon}_2, \ldots, \boldsymbol{\epsilon}_p\).
\end{itemize}
Mô hình phân tích nhân tố được mô tả:
\begin{eqnarray}
X_1 - \mu_1 &= \ell_{11}F_1 + \ell_{12}F_2 + \ldots + \ell_{1m}F_m + \boldsymbol{\epsilon}_1\\
X_2 - \mu_2 &= \ell_{21}F_1 + \ell_{22}F_2 + \ldots + \ell_{2m}F_m + \boldsymbol{\epsilon}_2
\end{eqnarray}
\[\vdots\]
\begin{eqnarray}
X_p - \mu_p &= \ell_{p1}F_1 + \ell_{p2}F_2 + \ldots + \ell_{pm}F_m + \boldsymbol{\epsilon}_p
\end{eqnarray}
hoặc, dưới dạng ký hiệu ma trận:
\begin{eqnarray}
\mathbf{X} - \boldsymbol{\mu} = \mathbf{L}\mathbf{F} + \boldsymbol{\varepsilon}
\end{eqnarray}
\end{frame}

\begin{frame}{Mô hình nhân tố trực giao (tiếp)}
Trong đó: \(\ell_{ij}\) được gọi là hệ số tải (loading) của biến thứ \(i\) trên nhân tố thứ \(j\), do đó ma trận \(L\) là ma trận của các hệ số tải nhân tố (matrix of factor loadings).

Lưu ý: 
\begin{itemize}
    \item Sai số thứ \(i\) \(\boldsymbol{\epsilon}_i\) chỉ liên quan đến phản hồi \(X_i\).
    \item \(p\) độ lệch : \(X_1 - \mu_1, X_2 - \mu_2, \ldots, X_p - \mu_p\) được biểu diễn bằng \(p + m\) theo các biến ngẫu nhiên  \(F_1, F_2, \ldots, F_m, \boldsymbol{\epsilon}_1, \boldsymbol{\epsilon}_2, \ldots, \boldsymbol{\epsilon}_p\), với \(F_i\) và \(\boldsymbol{\epsilon}_i\) không thể quan sát được.
\end{itemize}
Điều này phân biệt mô hình nhân tố trong (9-2) với mô hình hồi quy đa biến trong (7-23), trong đó biến độc lập [vị trí \(X_1, X_2, \ldots, X_p\)] có thể được quan sát được.
\end{frame}

\end{document}
